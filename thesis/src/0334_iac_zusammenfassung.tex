\subsubsection{Zusammenfassung}\label{iac_zusammenfassung}

Terraform ist das einzige Tool, welches explizit für die Provisionierung von Infrastruktur vorgesehen ist.
Durch die zahlreichen Provider bedeutet Provisionierung hier aber auch, dass Infrastruktur damit konfiguriert werden kann.
Es gibt also Provider für die Verwaltung von Datenbanken, Logging und anderer Software. \\

Von den untersuchten Tools verfolgt Terraform am konsequentesten den Ansatz: \textit{Treat your Servers as Cattle, not as Pets.}
Durch den deklarativen Ansatz sind Terraform Skripte gut lesbar und wiederholbar. \\

Bei den anderen Tools ist die Verwaltung von Infrastruktur ebenfalls möglich.
Für die Provisionierung von Cloud Ressourcen scheint Terraform aber mit Abstand das beliebteste IaC Tool zu sein. \\

Terraform wird hierbei häufig mit Ansible kombiniert.
Die Infrastruktur wird also mit Terraform bereitgestellt, um dann mit Ansible konfiguriert zu werden.


%Declarative programming is when you expresses the logic of a computation (the what) without
%describing the control flow (the how). Instead of writing step-by-step instructions, you simply
%describe what you want, and it will get done.
%
%
%Why is immutability better than mutability? Because it’s is easier to reason about. You don’t have
%to worry about things changing or leaving behind artifacts between deployments
%
%I am of the opinion that Terraform should not be combined with configuration management
%technologies, mainly because of their clashing philosophies.




















