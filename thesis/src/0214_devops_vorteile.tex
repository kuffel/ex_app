\subsubsection{Vorteile}\label{devops_vorteile}

Durch die Einführung von DevOps erhalten Unternehmen einerseits Wettbewerbsvorteile und andererseits aber auch Verbesserungen für Kunden und Mitarbeiter.
\footnote{König/Kugel, vgl.~\cite{Konig2019}~[S.293]}
\footnote{Buchan et. al, vgl.~\cite{Senapathi2018}~[S.7-8]}

\begin{itemize}
    \item \textbf{Time-To-Market:}
    Diese Zeit kann verkürzt werden.
    Die Unternehmen erhalten Wettbewerbsvorteile, da sie schneller auf Mitbewerber oder neue Kundenanforderungen reagieren können. \footnote{Lichtenberger, vgl.~\cite{Lichtenberger2017}~[S.245]}
    Produktideen können zügiger ausprobiert werden, um so bestimmte Hypothesen schneller prüfen zu können.

    \item \textbf{Erhöhte Stabilität:}
    Geringere Fehleranfälligkeit durch kleinere Releases und erhöhte Automatisierung. \footnote{Lichtenberger, vgl.~\cite{Lichtenberger2017}~[S.246]}
    Fehler können kostengünstiger behoben werden, da sie zeitnah entdeckt und Systeme schneller aktualisiert werden können.

    \item \textbf{Konflikte:}
    Überwindung von Konflikten zwischen den Abteilungen durch die Beseitigung der \textsl{Wall of Confusion}.
    Hand offs zwischen den Abteilungen werden minimiert oder sogar entfernt.
    Die Teams sind kleiner und selbstorganisiert, dadurch können sie selbstständiger arbeiten. \footnote{Lichtenberger, vgl.~\cite{Lichtenberger2017}~[S.246]}
    Formale Übergaben entfallen durch Tests und Automatisierung \footnote{Kasteleiner/Schwartz, vgl.~\cite{Kasteleiner2019}~[S.211-214}

    \item \textbf{Change Management:}
    Durch den Einsatz von Infrastructure as Code sind Änderungen besser nachvollziehbar und dokumentiert.
    Die Verwendung von Ticketsystemen, Scrum und IaC verbessert das Change Management. \footnote{Puppet State of DevOps Report 2020, vgl.~\cite{PUPPET}~[S.24 - S.26]}

    \item \textbf{Arbeitsbedingungen:}
    Mitarbeiter sind motivierter, da die Arbeitsbedingungen kontinuierlich optimiert werden.
    Entwickler können innovativer sein und selbstständig Ideen ausprobieren, die dem Kunden schneller einen Mehrwert bieten.
\end{itemize}


Laut Lichtenberger gilt oft der Grundsatz:

\begin{quotation}
    \textsl{Nicht der große Fisch frisst den kleinen, sondern der schnelle den langsamen.}
    \footnote{Lichtenberger, vgl.~\cite{Lichtenberger2017}~[S.248]}
\end{quotation}

Dieser Grundsatz zeigt, warum auch große Unternehmen an einer Beschleunigung ihrer Prozesse bei der Entwicklung und dem Betrieb von Software interessiert sein sollten. \\

Die australische Hotelbuchungswebsite Wotif hat z.B. durch die Einführung von DevOps die Releasezyklen von zwei Wochen auf einen Tag verkürzt und den Prozentsatz von Problemen,
die den Kunden betreffen, von 15\% auf 7\% gesenkt.
\footnote{Callanan, vgl.~\cite{Callanan2016}~[S.57]}


%    [RESTE]
%    - SlipWay: Team Moral erhöht `(Callanan, 59)`
%    - Übergeordnete Ziele von DevOps `(König/Kugel, P. 293)`
%    - Überwindung des chronischen IT-Konflikts - Wall of Confusion einreißen
%    - Häufige SW Releases erleichtern und optimieren
%    - Kontiniuierliche Verbesserung der Arbeitsbedingungen
%    - Erhöhung von Stabilität und Sicherheit
%    - Weg von der Idee bis zum Kunden optimieren
%    - Benefits from study of `(Buchan, 7-8)`:
%    - Teams happier and more engaged
%    - More frequent releases, smaller releases
%    - DevOps umgesetzt: Doppelt profitieren (Kundenanforderungen/ Motivation Mitarbeiter) `(Schwarz/Kasteleiner)`
%    - Produktideen sind Hypothesen `(Schwarz/Kasteleiner)`
%    - “It just means you are not relying on other teams to do the infrastructure. You have control over it – choice of tool to use for example. To get the feel of small startups in a big organisation” [Developer. `(Buchan, 6)`
%    - Product teams felt more valued in the new DevOps way of functioning. The embedded ops did not feel that they were just sitting in the dark maintaining servers and databases, but could see the value and impact of their work on real clients `(Buchan, 7)`