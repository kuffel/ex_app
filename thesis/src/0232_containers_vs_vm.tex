\subsubsection{Container vs VM}\label{containers_vs_vm}

Bei der Virtualisierung von Servern wird Hardware emuliert, um mehrere virtuelle Maschinen auf einem physikalischen Server auszuführen.
Diese Technologie ermöglicht eine bessere Ausnutzung von Hardware Ressourcen und wird in der IT-Industrie oft verwendet.
\footnote{Potdar et al., vgl.~\cite{Potdar2020}~[S.1419]}
Container stellen den nächsten Evolutionsschritt im Bereich der Virtualisierung dar.
Die Nachteile von Virtualisierung haben zu der Entwicklung von Containern geführt.
\footnote{Potdar et al., vgl.~\cite{Potdar2020}~[S.1419]}
Auch wenn VMs und Container verschiedene Probleme lösen sollen, so kann man diese folgendermaßen miteinander vergleichen:
\footnote{Potdar et al., vgl.~\cite{Potdar2020}~[S.1422]} \\

% https://www.tablesgenerator.com/

\begin{table}[H]
\centering
\begin{tabular}{|r|c|c|}
\hline
\multicolumn{1}{|c|}{} & \textbf{VM} & \textbf{Container} \\ \hline
\textbf{Isolation der Prozesse} & Auf Hardware Level & Auf Betriebssystem Level \\ \hline
\textbf{Betriebssystem} & Getrennt & Geteilt \\ \hline
\textbf{Startzeit} & Lang & Kurz \\ \hline
\textbf{Ressourcenverbrauch} & Mehr & Weniger \\ \hline
\textbf{Vorgefertigte Images} & Kaum verfügbar & Docker Registry \\ \hline
\textbf{Anpassbare Images} & Komplizierte Erstellung & Dockerfile \\ \hline
\textbf{Größe} & Größer (mit Betriebssystem) & Klein (nur Anwendung) \\ \hline
\textbf{Mobilität} & \begin{tabular}[c]{@{}c@{}}Einfach von einem Host zum \\ verschiebbar\end{tabular} & \begin{tabular}[c]{@{}c@{}}Container werden gelöscht\\ und neu erstellt.\end{tabular} \\ \hline
\textbf{Erstellungszeit} & Minuten & Sekunden \\ \hline
\end{tabular}
\caption{Container vs. VMs} \footnote{Pahl, vgl.~\cite{Pahl2015}~[S.25]}
\label{tab:container-vs-vm}
\end{table}

Tests haben gezeigt, dass Container in den meisten Fällen schneller sind als virtuelle Maschinen.
Container sind somit effizienter. \footnote{Potdar et al., vgl.~\cite{Potdar2020}~[S.1428]}
