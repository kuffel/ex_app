\subsection{Vorbereitung Entwicklungsumgebung}\label{implementierung_entwicklungsumgebung}

\subsubsection{Installation}

Um die Beispielanwendung erstellen und ausführen zu können, müssen auf dem PC einige Softwarepakete installiert werden.
Dieser Abschnitt beschreibt welche Komponenten installiert werden müssen um die nachfolgenden Abschnitte ausführen zu können.

Die Beschreibungen setzen voraus das ein Linux Betriebssystem verwendet wird.
Auf Debian basierenden Betriebssystemen sollten die folgenden Befehle alle notwendigen Pakete installieren:

\paragraph{Git}

Zur Versionierung wird Git verwendet, dieses muss zunächst installiert und konfiguriert werden:

\lstset{language=bash}
\begin{lstlisting}[frame=htrbl, caption={Git Installation}, label={lst:git_setup}]


asdf plugin add terraform
asdf install terraform 0.13.5
asdf global terraform 0.13.5
\end{lstlisting}

\paragraph{Docker}

Je nach Betriebssystem kann die Installation abweichen, sollte aber generell mit den folgenden Kommandos möglich sein.




\paragraph{Erlang und Elixir}

Zur Installation von Erlang und Elixir kann ASDF verwendet werden. `(https://github.com/asdf-vm/asdf)`

.tool-versions Erklären

\paragraph{AWS CLI}

Die AWS CLI wird im Hintergrund von Terraform verwendet und muss installiert und konfiguriert werden:

\paragraph{Terraform}

Terraform lässt sich ebenfalls über ASDF installieren:
