\paragraph{AWS}\label{caas_tools_aws}

AWS ist eine Tochterfirma von Amazon, die im Jahre 2003 gestartet wurde um den immer größer werdenden Bedarf an Infrastruktur für den Amazon Online Store bereitzustellen.
\footnote{{EC2 Origins, vgl.~\cite{AWS_EC2_ORIGINS}}}

Nachdem dieses Vorhaben erfolgreich umgesetzt wurde, erkannte Amazon, dass dieses Angebot auch für anderen Unternehmen interessant sein könnte.
\footnote{{Informationen zu AWS, vgl.~\cite{AWS_ABOUT}}}

Zu den beliebtesten Amazon Diensten gehören EC2 für virtuelle Maschinen, S3 für Storage und Lambda
\footnote{{Top 10 AWS Services according to popularity, vgl.~\cite{AWS_MEDIUM_TOP10_SERVICES}}}

AWS ist im Cloud Computing Bereich mit Angeboten von IaaS, PaaS bis hin zu SaaS vertreten.
AWS ist der Marktführer im Bereich Public Cloud und weist mit einem Marktanteil von 33\% in Q4 2019 einen größeren Marktanteil als Azure (18\%) und Google Cloud (8\%) zusammen auf.
\footnote{{Amazon ist die Nummer 1 in der Cloud, vgl.~\cite{AWS_STATISTA}}}

Der jährliche Umsatz von AWS hat sich seit 2013 mit 3 Milliarden Dollar auf 35 Milliarden Dollar im Jahre 2019 mehr als verzehnfacht.
\footnote{{Annual revenue of Amazon Web Services from 2013 to 2019, vgl.~\cite{AWS_STATISTA_REVNUE}}} \\

AWS hat den größten Umfang aller Cloud Anbieter.
Dieser Abschnitt erläutert die wichtigsten Begriffe, die für die Einführung in AWS notwendig sind:
% `(https://www.datamation.com/cloud-computing/aws-vs-azure-vs-google-cloud-comparison.html)`

\begin{itemize}
    \item \textbf{Region und Availability Zone:}
    AWS betreibt weltweit zur Zeit 24 Regionen.
    Diese Regionen werden von mehreren geografisch getrennten Rechenzentren (Availability Zone (AZ)) zur Verfügung gestellt.
    Jede AZ hat eine eigene Stromversorgung, Vernetzung und Konnektivität in einer AWS Region.
    Die Kommunikation innerhalb der AZs einer Region erfolgt per Glasfaser und hat eine niedrige Latenz.
    AZs werden so getrennt (> 100 km), dass Naturereignisse nur eine AZ in einer einzelnen Region betreffen sollten.
    \footnote{{Regionen und Availability Zones, vgl.~\cite{AWS_REGIONS_AND_AZS}}}

    \item \textbf{Elastic Compute Cloud (EC2):}
    Dieser Dienst stellt virtuelle Maschinen zur Verfügung, die über eine API oder eine Management-Konsole gestartet werden können.
    Dabei stehen über 350 verschiedene Größen zur Verfügung.
    Diese unterscheiden sich bei der Wahl der CPU, RAM, IO und Netzwerkperformance.
    % - Ondemand Instances: Können einfach gestartet werden und werden je angefangener Stunde abgerechnet
    % - Reserved Instances: Reservierungen für 1,2, oder 3 Jahre für einen bestimmten Typ oder kleiner.
    % - Spot Instances: Börse mit übriggebliebener Rechenzeit von Instanzen, die vor Ablauf einer vollen Stunde beendet werden, bis zu 90\% Kostenersparnis.
    \footnote{{Amazon EC2, vgl.~\cite{AWS_EC2}}}

    \item \textbf{Identity and Access Management:}
    Authentifizierung und Autorisierung von Benutzern und Services auf AWS. Alle AWS Ressourcen können über IAM abgesichert werden.
    \footnote{{AWS Identity and Access Management (IAM), vgl.~\cite{AWS_IAM}}}

    \item \textbf{Elastic Container Registry (ECR):}
    Von AWS verwaltete Container Registry, Zugriffsregeln lassen sich per IAM einstellen.
    \footnote{{Amazon Elastic Container Registry, vgl.~\cite{AWS_ECR}}}

    \item \textbf{Elastic Load Balancing (ELB):}
    Von AWS verwalteter Load-Balancer in unterschiedlichen Ausprägungen.
    Der \textbf{Application Load-Balancer (ALB)} ist für HTTP Dienste und unterstützt Weiterleitungen und HTTP Healthchecks.
    Mit dem \textbf{Network Load-Balancer (NLB/Classic)} können alle anderen TCP/UDP Dienste verwendet werden.
    \footnote{{Elastic Load Balancing, vgl.~\cite{AWS_ELB}}}

    \item \textbf{Virtual Private Cloud (VPC):}
    Erlaubt die Erstellung von isolierten Netzwerken in der AWS Cloud.
    Dabei können auch VPNs für die Einbindung von Unternehmensnetzwerken und Security Groups zur Verwaltung von Firewallregeln verwendet werden.
    \footnote{{Amazon Virtual Private Cloud, vgl.~\cite{AWS_VPC}}}

\end{itemize}

% \textbf{Container as a Service Angebote} \\
% ###### 3.4.2.1.2. Container as a Service Angebote
% `(https://www.dragonspears.com/blog/aws-container-orchestration-101-ecs-vs-fargate-vs-eks)`

AWS bietet grundsätzlich zwei verschiedene Varianten für den Betrieb von Containern as a Service (CaaS): \\

Der eigene Dienst ist dabei der \textbf{Elastic Container Service (ECS)}.
Hier besteht eine gute Integration in die anderen AWS Dienste wie EC2, IAM und Cloudwatch.
Im \textit{ECS Traditional} Modus läuft der ECS Client auf vom Kunden verwalteten EC2 Instanzen.
Über ECS können dort dann Tasks und Services gestartet werden.
Zusätzlich erlaubt ECS auch die Nutzung von Fargate.
Dabei handelt es sich um einen Container as a Service Dienst bei dem die Verwaltung eigener Infrastruktur entfällt.
\footnote{{Amazon Elastic Container Service, vgl.~\cite{AWS_ECS}}} \\

Zusätzlich bietet AWS auch noch den \textbf{Elastic Kubernetes Service (EKS)} an.
Dabei handelt es sich um einen von AWS verwalteten Kubernetes Dienst.
Dieser ist vollständig mit Kubernetes kompatibel und kann über ein Amazon EKS Dashboard verwaltet werden.
Auch EKS erlaubt die Verwendung von Fargate zur serverlosen Ausführung von Kubernetes Pods.
\footnote{{Amazon Elastic Kubernetes Service, vgl.~\cite{AWS_EKS}}} \\

Beide Dienste stehen in allen Regionen zur Verfügung und erlauben die Nutzung von EC2 Reserved und Spot Instances.