\newpage
\subsection{Circle CI}\label{implementierung_circle_ci}

Dieser Abschnitt beschreibt die Einrichtung und Konfiguration der Circle CI-Pipeline.
Die Pipeline wird bei jedem Pull Request auf GitHub gestartet.
Bei jedem akzeptierten Pull Request für den Master wird die gleiche Pipeline durchgeführt.

\subsubsection{Grundlagen}

Circle CI verwendet für die Beispiele vereinfachte Konfigurationen.
Für komplexere Abläufe lassen sich Konfigurationen für die Version 2.1. erstellen.
Diese haben folgende Unterteilung und erlauben die Wiederverwendung von einzelnen Schritten.
\footnote{{Orbs, Jobs, Steps, and Workflows, vgl.~\cite{CIRCLE_ORBS_JOBS_STEPS_WORKFLOWS}}}

\begin{itemize}
  \item \textbf{Executors:}
  Durch Executors lässt sich die Ausführungsumgebung in Circle CI bestimmen.
  Dabei können verschiedene Executors für unterschiedliche Schritte verwenden werden.
  Der Docker Executor startet das gewünschte Docker Image und führt alle Schritte innerhalb dieses Containers durch.
  Ein Machine Executor kann Linux, Windows oder MacOS verwenden und führt die Schritte direkt auf einer VM durch.
  \footnote{{Executors and Images, vgl.~\cite{CIRCLE_EXECUTORS_INTRO}}}

  \item \textbf{Commands:}
  Mit Commands lassen sich Abfolgen von Schritten für die Wiederverwendung gruppieren.
  Ein Command wird benannt und lässt sich in den Jobs immer wieder aufrufen.

  \item \textbf{Jobs:}
  Jobs sind benannte Abfolgen von eingebauten Commands und selbst definierten Commands.
  Hierbei wird auch der Executor festgelegt, d.h. die Umgebung in der ein Job ausgeführt werden soll.

  \item \textbf{Workflows:}
  Definiert eine Abfolge von Jobs.
  Hierbei können Bedingungen, wie der aktuelle Branch, definiert werden.

  \item \textbf{Orbs:}
  Bei Orbs handelt es sich um wiederverwendbare Konfiguration für Commands, Jobs und Workflows.
  In einer CI-Pipeline muss immer wieder Software auf den Executors installiert und konfiguriert werden.
  Die AWS CLI erfordert z.B. die Installation von Python und die Konfiguration von AWS Credentials.
  Wenn mehrere Schritte mit unterschiedlichen Executors die CLI brauchen entsteht Redundanz.
  Orbs verwenden Environment Variablen und erlauben so eine einfache Konfiguration von Jobs, die Abhängigkeiten zu Software wie z.B. AWS CLI und Terraform haben.
\end{itemize}

\subsubsection{Einrichtung}

Um Circle CI zu verwenden muss zunächst ein kostenloser Account angelegt werden.
Dafür kann man sich direkt mit seinem GitHub Account registrieren.
Nachdem man Circle CI die entsprechenden Berechtigungen im Github Account eingeräumt hat, erscheint eine Liste aller Projekte, die im eigenen GitHub Account verfügbar sind.

Wenn man ein Projekt mit \textbf{Set Up Project} auswählt, kann direkt ein neuer Commit mit der Circle CI Konfiguration erstellt werden.
Über ein Dropdown kann hier eine vorgefertigte Konfiguration für die verwendete Programmiersprache ausgewählt werden.

Nachdem man die Änderungen mit \textbf{Commit and Run} bestätigt, erzeugt Circle CI einen Commit und eine erste Pipeline wird gestartet.

\paragraph{Executors und Orbs}

Die Pipeline für die Beispielanwendung verwendet zwei Arten von Executors:

\begin{itemize}
  \item \textbf{Docker Executor:}
  Verwendet ein Elixir Docker Image für die Schritte, die eine Elixir Umgebung benötigen.

  \item \textbf{Machine Executor:}
  CircleCI Linux Maschine für Terraform, Docker und AWS CLI Operationen.
\end{itemize}

Die Beispielanwendung benötigt die AWS CLI, um Images in AWS ECR veröffentlichen zu können.
Dafür gibt es zwei offizielle Orbs von CircleCI:

\begin{itemize}
  \item \textbf{AWS CLI}
  Dieser Orb installiert und konfiguriert die AWS CLI in einem Machine Executor.
  Die Konfiguration der AWS Access Keys erfolgt dabei über die Umgebungsvariablen, die über die CircleCI Benutzeroberfläche konfiguriert werden.
  Dieser Orb kann einfach mit \texttt{aws-cli/setup} in den Steps verwendet werden.
  In den darauffolgenden Schritten steht dann die AWS CLI zur Verfügung.
  \footnote{{Orbs AWS CLI, vgl.~\cite{CIRCLE_ORBS_AWS_CLI}}}

  \item \textbf{AWS ECR}
  Um Images in AWS ECR publizieren zu können, muss der Docker Client sich an der Registry anmelden.
  Dieser Orb ermöglicht den Docker Login an ECR mit dem Schritt \texttt{aws-ecr/ecr-login}
  \footnote{{Orbs AWS ECR, vgl.~\cite{CIRCLE_ORBS_AWS_ECR}}}
\end{itemize}

Die Circle CI Konfiguration wird unter \hyperref[lst:circle_orbs_and_executors]{.circleci/config.yml} abgelegt.

\subsubsection{Commands}

Für die Pipeline wurden sich ständig wiederholende Schritte als Commands definiert.
Diese werden dann in den jeweiligen Jobs wiederverwendet.

\begin{itemize}
  \item \textbf{setup\_elixir:}
  Dieses Kommando richtet zunächst die Elixir und Erlang Paketmanager Hex und Rebar ein.
  Die Abhängigkeiten werden kompiliert und zusammen mit einer Prüfsumme über die Datei mix.lock im Cache abgelegt.
  Falls die Abhängigkeiten bereits kompiliert wurden, können sie bei einer Wiederholung dieses Kommandos aus dem Cache geladen werden.
  Das vollständige Kommando findet sich im Anhang unter \hyperref[lst:circle_command_setup_elixir]{Circle CI - Command - setup\_elixir}.

  \item \textbf{setup\_terraform:}
  Terraform wird für das Deployment verwendet.
  Die CLI wird verwendet, um Terraform Dateien zu validieren, Umgebungen zu löschen und zu erstellen.
  Dieses Kommando lädt die CLI herunter und kopiert diese in das entsprechende Linux Verzeichnis.
  Das vollständige Kommando befindet sich im Anhang unter \hyperref[lst:circle_command_setup_terraform]{Circle CI - Command - setup\_terraform}.

\end{itemize}

\subsubsection{Jobs}

Jobs sind eine Abfolge von Kommandos, die später in einem Workflow eingesetzt werden.
Jeder Job definiert somit die Schritte und den Executor auf dem dieser ausgeführt werden soll.
Dabei ist zu beachten, dass Jobs immer mit einem neuen Executor gestartet werden.
Das erlaubt die Parallelisierung.
Es erfordert jedoch auch immer die Initialisierung der Umgebung.

\paragraph{Statische Codeanalyse}

Dieser \hyperref[lst:circle_job_check_code]{Schritt} überprüft die Formatierung mit \texttt{mix format} und den Code statisch nach den Regeln von Credo mit der Ausführung von \texttt{mix credo}.
Schlägt einer dieser beiden Schritte fehl, wird die gesamte CI/CD Pipeline als fehlgeschlagen markiert.
Der Ersteller des Pull Requests erhält ein frühes Feedback darüber, dass der Code nicht korrekt formatiert ist oder nicht den festgelegten Regeln enspricht.

\paragraph{Kompilierung}

In diesem \hyperref[lst:circle_job_build]{Schritt} werden die Quellen des Projekts kompiliert.
Dabei wird das Kommando \texttt{mix compile --force --warnings-as-errors} verwendet.
Wenn Warnungen bei der Kompilierung auftreten, wird die CI/CD Pipeline als fehlgeschlagen markiert.
Der Parameter \texttt{--warnings-as-errors} soll dazu dienen, nicht verwendete Variablen früh im Buildprozess zu erkennen und den Entwickler darauf hinzuweisen.

\paragraph{Tests}

Bei diesem \hyperref[lst:circle_job_test]{Schritt} werden die Unit Tests der Anwendung mit \texttt{mix test} ausgeführt.
Wenn Tests fehlschlagen wird die gesamte Pipeline als fehlgeschlagen markiert und beendet.
Die Tests verwenden coverall, um zu ermitteln welche Testabdeckung erreicht wurde.
Wird die gewünschte Abdeckung nicht erreicht, so wird die Pipeline als Fehlschlag beendet.

\paragraph{Dokumentation}

Bei diesem \hyperref[lst:circle_job_documentation]{Schritt} wird die Dokumentation mit \texttt{mix test} erstellt.
Die erstellte ZIP-Datei wird im Anschluss unter den Build Artefakten in Circle CI abgelegt.
Die Dokumentation kann nach einem Build über die Oberfläche von Circle CI heruntergeladen werden.

\paragraph{Docker}

Zur Erstellung der Docker Container wird zunächst die Versionsnummer in der Anwendung aktualisiert.
Dazu wird ein \hyperref[lst:projekt_version_sh]{Skript} aufgerufen, das die aktuelle Semantic Version mit dem Git Commit Hash ergänzt.
Anschließend wird das Docker Image durch Aufruf von \texttt{make docker} erstellt.

Dieses Image muss getaggt und in AWS ECR veröffentlicht werden.
Dazu wird der AWS CLI und AWS ECR Orb verwendet.
Als Docker Tag wird hier ebenfalls die Version und der Git Commit Hash verwendet.
Da Docker Tags kein Pluszeichen zur Trennung von Version und Hash unterstützen, muss mit einem Bindestrich getrennt werden. \\

Abschließend wird das Image in das AWS ECR Repository gepusht und steht somit für ein Deployment zur Verfügung.
Die YAML Definition des \hyperref[lst:circle_job_docker]{Schritts} befindet sich im Anhang.

\paragraph{AWS Lambda Funktion}

Nicht alle Teile der CI-Pipeline lassen sich einfach in Orbs oder Shell Skripten implementieren.
Gerade Schritte die REST-APIs von AWS und GitHub verwenden ist der Einsatz von Python empfehlenswert. \\

Es wurde AWS Chalice verwendet.
Es handelt sich dabei um ein Python Framework mit dem serverlose AWS Lambda Funktionen erstellt werden können. \footnote{{AWS Chalice, vgl.~\cite{AWS_CHALICE}}}
Diese Funktionen lassen sich über eine REST API verwenden.
Somit wird vermieden in der CI-Pipeline Python Skript auszuführen. \\

Diese \hyperref[lst:lambda_app_py]{REST Endpunkte} wurden dafür implementiert:

\begin{itemize}
  \item \texttt{POST /cleanup\_images:}
  Verwendet die Python AWS-Library, um zu ermitteln welche ECR Images gerade noch in einem ECS-Cluster verwendet werden.
  Nicht verwendete Images werden automatisch gelöscht.
  Dieses Vorgehen spart Kosten für die Speicherung von nicht verwendeten Images in ECR.

  \item \texttt{GET /pull-requests:}
  Gibt eine Liste aller Pull Requests in dem ex\_app Github Projekt zurück.

  \item \texttt{GET /unused-workspaces:}
  Ermittelt anhand bereits geschlossener Pull Requests und der Terraform Statefiles im S3 Bucket welche Deployments nicht mehr benötigt werden.
  Diese Deployments und Workspaces können im Rahmen eines Master Build Workflows entfernt werden.

  \item \texttt{POST /add-comment/{PULL\_REQUEST\_ID}:}
  Erzeugt einen neuen Kommentar in dem GitHub Pull Request.
  Dieser Endpunkt wird dazu verwendet, den Kommentar mit der URL zu der Vorschauumgebung zu erzeugen.
\end{itemize}

Die jeweiligen Endpunkte werden mit Bash Skripten aus der CI/CD Pipeline heraus aufgerufen.
Diese Bash Skripte können einfach mit dem in Linux integrierten Kommando \texttt{wget} implementiert werden.

\paragraph{Vorschauumgebung installieren}

Damit der Reviewer des Pull Requests neben den Code-Änderungen die Änderungen auch direkt in der Anwendung anschauen kann, wird eine \hyperref[lst:circle_job_deploy_preview]{Vorschauumgebung installiert}.
Dadurch werden die Terraform Skripte verifiziert und ein Reviewer muss kein eigenes Deployment durchführen. \\

In dem Job wird zunächst die AWS CLI konfiguriert, Terraform installiert und die Terraform Dateien validiert.
Um die Vorschauumgebung zu installieren, wird ein eigener Terraform Workspace erstellt.
Dieser Workspace bekommt die ID des Pull Requests mit dem Prefix \texttt{preview\_}. \\

Danach wird die Infrastruktur mit \texttt{terraform apply} und unter Verwendung der entsprechenden Variablen installiert.
Nachdem \texttt{terraform apply} durchgeführt wurde, kann es einige Minuten dauern bis das Deployment tatsächlich bereitsteht. \\

Um zu erkennen ob die gewünschte Version der Anwendung installiert ist wurden zwei zusätzliche HTTP Response Header eingefügt.
Diese enthalten in einem \texttt{x-app-version} Header die Versionsnummer und in einem \texttt{x-app-build} Header den Git Commit Hash in jeder HTTP-Antwort. \\

Das Skript \hyperref[lst:scripts_wait_for_deployment]{\texttt{wait\_for\_deployment.sh}} bekommt die geplante URL zur Vorschauanwendung und den erwartetet Git Commit Hash als Parameter.
Damit ruft das Skript periodisch die URL auf und prüft, ob eine Antwort mit dem erwarteten Header zurückkommt.
Zwischen den Versuchen wartet das Skript 10 Sekunden und nach maximal 10 Minuten wird das Deployment als fehlgeschlagen markiert.

Wenn der Header erfolgreich erkannt wurde, wird über die AWS Lambda Funktion und das Skript \hyperref[lst:scripts_add_pr_comment]{\texttt{add\_pr\_comment.sh}}
ein GitHub Kommentar mit der Vorschau URL in dem Pull Request erstellt.

\paragraph{Vorschauumgebungen entfernen}

Nachdem eine Vorschauumgebung installiert worden ist, unterscheidet sich diese für AWS nicht von einem regulären Deployment.
Diese Umgebung würde also dauerhaft bestehen bleiben und so Kosten erzeugen.
Nachdem der Pull Request akzeptiert oder geschlossen wurde, kann die Umgebung entfernt werden. \\

Dieser \hyperref[lst:circle_job_remove_preview_deployments]{Schritt} in der Circle CI-Pipeline soll das sicherstellen
und verwendet dafür die Lambda Funktion.
Es soll ermittelt werden welche Workspaces gelöscht werden können.
Dazu werden die Terraform Workspaces ermittelt bei denen kein offener Pull Request mehr vorliegt.
Diese Workspaces werden folglich nicht mehr benötigt und werden in diesem Schritt automatisch entfernt.

\paragraph{Installation in Produktivumgebung}

In diesem \hyperref[lst:circle_job_deploy_production]{Schritt} wird die Anwendung in der Produktivumgebung ausgerollt.
Nach jedem angenommenen Pull Request oder einer Änderung auf dem Master Branch wird die Produktivumgebung aktualisiert.
Dafür wird der \texttt{production} Workspace in Terraform verwendet um die Images die im Rahmen der Pipeline für den Master Branch entstehen zu installieren. \\

Das Installieren der Anwendung im Produktivsystem unterscheidet sich nur geringfügig von der Installation einer Vorschauumgebung.
Der Circle CI Job verwendet die gleichen Schritte mit abgeänderten Parametern für das Deployment. \\

Nachdem der Master erfolgreich deployed wurde, werden durch einen Aufruf an die AWS Lambda Funktion nicht mehr verwendete ECR Images gelöscht.

\subsubsection{Workflow}

Die zuvor implementierten Schritte werden in einem Workflow in Circle CI abgebildet.
Mit Workflows lässt sich festlegen, welche Jobs in welcher Reihenfolge und unter welchen Bedingungen ausgeführt werden sollen. \\

Grundsätzlich würde Circle CI die Jobs parallel ausführen.
Durch die Angabe der Abhängigkeiten mit dem Schlüsselwort \texttt{requires} kann eine Reihenfolge definiert werden.
Jeder Job unterstützt zusätzlich die Angabe eines Filters.
Damit kann gesteuert werden, ob ein Job nur auf einem bestimmten Branch durchgeführt werden soll.

\begin{itemize}
  \item \textbf{Statische Codeanalyse:}
  Dieser Schritt wird immer als erstes und auf allen Branches ausgeführt.

  \item \textbf{Kompilierung:}
  Dieser Schritt wird immer nach der statischen Codeanalyse ausgeführt.

  \item \textbf{Tests:}
  Nachdem die Kompilierung erfolgreich war, werden die Tests gestartet.

  \item \textbf{Dokumentation:}
  Die Erstellung der Dokumentation startet parallel, nachdem die Kompilierung und die Tests erfolgreich waren.

  \item \textbf{Docker:}
  Nach erfolgreichen Tests wird eine Docker Image generiert und in ECR veröffentlicht.

  \item \textbf{Vorschauumgebung installieren:}
  Dieser Schritt wird nur durchgeführt, wenn der Workflow nicht auf dem Master Branch durchgeführt wird.

  \item \textbf{Vorschauumgebungen entfernen:}
  Entfernt die Vorschauumgebungen, wenn der Master-Branch im aktuellen Workflow gebaut wird.

  \item \textbf{Installation in Produktivumgebung:}
  Wird nur auf dem Master Branch durchgeführt und nimmt das Deployment in die Produktivumgebung vor.
\end{itemize}

Auch nach einem akzeptierten Pull Request werden die Schritte, die für einen Pull Request bereits durchgeführt wurden erneut durchgeführt.
Das soll sicherstellen, dass der Master-Branch die gleichen Tests durchläuft und sich ein Deployment für die Vorschauumbgebung kaum von einem Produktivdeployment unterscheidet.
Bei einem nicht akzeptierten Pull Request wird die Vorschauumgebung spätestens beim nächsten Master-Branch Build entfernt. \\

Der komplette Ablauf ist im Anhang unter \hyperref[lst:circle_workflows]{Circle CI - Workflows} definiert.

