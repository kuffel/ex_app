\subsection{Vorgehen}\label{vorgehen}

Diese Arbeit erläutert zunächst die verschiedenen Begriffe wie DevOps, Docker, Infrastructure as Code und Continuous Delivery,
um anschließend gängige Software zur Erreichung von Continuous Delivery zu untersuchen. \\

Hierfür werden die gängigen Tools anhand der aktuellen Literatur ausgewählt und miteinader verglichen.
Bei der Auswahl der Tools wird darauf geachtet, dass diese sich bis zu einem gewissen Punkt kostenlos nutzen lassen und im Idealfall Open-Source sind. \\

Anschließend wird eine Continuous Delivery Pipeline auf Basis der ausgewählten Tool implementiert.
Software und Tools, die nicht den Betrieb von eigenen Servern erfordert und \textit{Cloud Native} sind, werden ebenfalls bevorzugt.