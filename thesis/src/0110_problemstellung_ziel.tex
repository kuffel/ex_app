\subsection{Problemstellung und Zielsetzung}\label{problemstellung_ziel}

Dem Begriff DevOps werden viele Definitionen zugeordnet.
DevOps beschreibt vor allem eine Philosophie, die von Unternehmen zunächst verstanden und angewendet werden muss.
Die Herausforderungen, die sich dabei ergeben, sind nicht nur technischer Natur und können nicht von einem auf den anderen Tag umgesetzt werden.
Neben den organisatorischen Maßnahmen gibt es darüber hinaus noch technische Fragestellungen, die bei einer \textsl{Umstellung} auf DevOps zu beachten sind.
Diese Arbeit erläutert die verschiedenen Begriffe, welche rund um \textsl{DevOps, Continuous Integration und Continuous Delivery} in der Literatur beschrieben werden. \\

Ziel ist eine Continuous Delivery Pipeline für Docker-Container basierte Anwendungen exemplarisch umzusetzen.
Die Umsetzung soll hierbei die zuvor ermittelten für DevOps relevanten organisatorischen Herausforderungen berücksichtigen
und es Entwicklern ermöglichen, Codeänderungen innerhalb weniger Minuten in einer Produktivumgebung zu installieren. \\

Hierzu werden gängige Tools im Zusammenhang mit Codeverwaltung, Zusammenarbeit der Entwickler, Build Systeme, Infrastructure as Code und Container
as a Service Angebote diverser Cloud Provider untersucht, um daraus einen möglichen Weg für Continuous Delivery zu implementieren.