\paragraph{Jenkins}\label{ci_services_tools_jenkins}

Im Jahre 2007 erschien Jenkins als Hudson Open-Source Build Server und wurde später umbenannt. \footnote{Hudson Announcement, vgl.~\cite{JENKINS_ANOUNCEMENT}}
Dieser wurde mit Abstand der beliebteste Build Server.
Durch zahlreiche Plugins (> 1.500 Stück) lässt er sich für alle Einsatzzwecke anpassen. \footnote{Jenkins Awards, vgl.~\cite{JENKINS_AWARDS}}
Der Dienst wird i.d.R. als selbst gehosteter Server eingesetzt und hat eine Master/Slave Architektur.
Der Master kontrolliert die sogenannten Jenkins Agents auf denen die Builds ausgeführt werden.
Es lassen sich jedoch auch Builds auf dem Master ausführen. \\

Durch die Unterstützung von Docker basierenden Agents lässt sich jede Programmiersprache - für die es ein Docker-Image gibt - verwenden.
Jenkins Agents lassen sich entweder über SSH oder über einen Java basierten Jenkins Agent verwalten.
Dadurch ist es möglich Windows, Linux und MacOS zu verwenden. \\

Die Konfiguration kann entweder über die Jenkins Oberfläche oder über Jenkinsfiles erstellt werden.
Ein Jenkinsfile wird dabei in der Programmiersprache Groovy verfasst. \footnote{Pipeline Syntax, vgl.~\cite{JENKINS_JENKINSFILE}}

Um Passwörter und andere Geheimnisse in die Builds zu injizieren stehen Plugins zur Verfügung.
Diese sorgen ebenfalls dafür, dass injizierte Geheimnisse nicht in den Log Dateien der Builds sichtbar sind. \\

Da es sich um einen selbst gehosteten Buildsserver handelt, lassen sich die Kosten nicht pauschal bestimmen.
Die Verwaltung einer Jenkins Infrastruktur erfordert also eine Administration vom Unternehmen selbst.
