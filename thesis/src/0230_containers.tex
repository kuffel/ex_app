\newpage
\subsection{Container}\label{container}

Bei der Verwendung von Containern sind häufig Docker-Container gemeint.
Es handelt sich dabei jedoch nicht um die einzige Implementierung von Containern.
Sie erlauben das Verpacken von Software mit ihren Abhängigkeiten in ein standardisiertes Format zur Weiterverteilung.
\footnote{What is a Container?, vgl.~\cite{DOCKER_WEBSITE}} \\

Bei der Einführung von Docker im Jahre 2013 wurde die bereits in Linux vorhandene Container Technologie einfacher für eine breite Masse von Entwicklern nutzbar gemacht.
Durch eingebaute Funktionen im Linux Kernel, wie LXC, Namespaces und Control groups, war es schon vor der Einführung von Docker möglich Container zu verwenden.
\footnote{Pahl, vgl.~\cite{Pahl2015}~[S.25 - S.26]} \\

Diese Vereinfachung der Nutzung von Containern führte zu einer verstärkten Adoption von Containern zur Veröffentlichung von Applikationen.

Mit \textsl{Containerd} existiert mittlerweile auch eine standardisierte Laufzeitumgebung für Container.
Docker und Kubernetes verwenden diese Laufzeitumgebung.
\footnote{Containerd, vgl.~\cite{CONTAINERD_WEBSITE}} \\

Docker ist mit Abstand die populärste Software zur Ausführung von Containern.
Die folgenden Abschnitte werden sich also vorwiegend auf Docker beziehen.
\footnote{What are the best Virtual Machine Platforms and Containers Tools?, vgl.~\cite{STACKSHARE_CONTAINERS_VMS}}


%    [RESTE]
%    - Container-based virtualization uses single kernel to run multiple instances on an operating system and virtualization layer runs as an application within the operating system `(Singh, 804)`
%    - Platform-as-a-service clouds can use containers to manage and orchestrate applications. `(Pahl, 24)`
%    - Server virtualization is a technological innovation broadly used in IT enterprises `(Potdar, 1419)`
%    - Although VMs and containers are both virtualization techniques, they solve different problems. `(Pahl, 24)`
%    - Recent OS advances have improved their multi- tenancy capabilities: LXC, Namespace isolation, Control groups. `(Pahl, 25 - 26)`
%    - High demand for low overhead virtualization technology, docker is one of them `(Potdar, 1419)`
%    - solution, containerization: lightweight portable runtime, capability to develop, test, and deploy appli- cations to a large number of servers, capability to interconnect containers `(Pahl, 24)`
%    - Drawbacks VMs: Large, unstable performance, boot up times, unable to solve difficulties like management, sw updates, and ci/cd `(Potdar, 1419)`
%    - VM Drawbacks led to containerization, uses the host os, no guest os, os kernels isolated space. `(Potdar, 1420)`
%    - virtualization technologies have de- veloped out of the need for scheduling processes as manageable container units `(Pahl, 25)`
%    - combines the application, related dependencies, and system libraries organized to build in the form of a container. `(Potdar, 1420)`
%    - Essential parts of docker: Docker daemon, REST API, Docker client . `(Potdar, 1420)`
%    - The Docker client and daemon can run on the same system or a Docker client can ccommunicate through sockets or RESTful API to a remote Docker daemon `(Singh, 806)`
%    - Images: Two ways to get them (Registry / Dockerfile) `(Potdar, 1421)`
%    - Docker images are read-only templates and Docker registries hold these images `(Singh, 806)`
%    - A container is represented by lightweight images; VMs are also based on images but full, monolithic ones `(Pahl, 26)`
%    - A container is a light weight operating system running inside the host system, running instructions native to the core CPU, eliminating the need for instruction level emulation or just in time compilation `(Raja, 610)`
%    - Container-based virtualization improves performance and efficiency compared to conventional hypervisor since additional resources needed for each OS is eliminated `(Singh, 805)`
%    - The containers are lightweight, portable, efficient and can run on physical servers. We can run more containers on a physical servers than virtual machines which results in higher resource utilization `(Singh, 805)`
%    - Containers are based on layers composed from indi- vidual images built on top of a base image that can be extended `(Pahl, 26)`
%    - Containers: Running application held in the container `(Potdar, 1421)`
%    - A container holds packaged, self-contained, ready-to-deploy parts of applications and, if neces- sary, middleware and business logic (in binaries and libraries) to run applications `(Pahl, 25)`
%    - Containers vs VMs: Docker sometimes referred as lightweight VMs `(Potdar, 1421)`
%    - Containers are a similar but more light- weight virtualization concept; they’re less resource and time-consuming, thus they’ve been suggested as a solution for more interoperable application pack- aging in the cloud `(Pahl, 24)`
%    - The repositories play a central role in providing access to possibly tens of thousands of reusable pri- vate and public container images `(Pahl, 26)`
%    - Network management is based on two methods for assigning ports on a host—network port mappings and container linking. `(Pahl, 27)`
%    - VMs vs Containers:  `(Potdar, 1422)`
%    - Isolation Process Level:  Hardware /  Operating System
%    - Operating System: Seperated / Shared
%    - Boot up time: Long / Short
%    - Ressource usage: More / Less
%    - Prebuilt images: Hard to find / Docker registry
%    - Customised preconfigured images: Hard to build / Easy to build
%    - Size: Bigger (contains OS) `(Pahl, 25)` / Smaller
%    - Mobility: Easy to move to a new host OS / Destroy and recreated
%    - Creation time: Minutes / Seconds
%    - It is observed that Docker containers perform better over VM in every test, as the presence of QEMU layer in the virtual machine makes it less efficient than Docker containers `(Potdar, 1428)`
%    - Containers as a service (CaaS) is a form of container-based virtualization in which container engines, orchestration & underlying compute resources are provided to users as a service. Most of the public cloud providers like Amazon Web Services (AWS), IBM, Google, Rackspace and Joyent have some type of CaaS offering `(Singh, 807)`
