\subsubsection{Vorteile}\label{iac_vorteile}

In der Literatur finden sich folgende positiven Aspekte für den Einsatz von IaC:
\footnote{Siebra et al., vgl.~\cite{Siebra2019}~[S.428]}
% `(Johann, 118-120)`
\footnote{Artac et al., vgl.~\cite{Artac2017}~[S.497]}

\begin{itemize}
    \item \textbf{Dokumentation:}
    Durch die IaC Skripte ist dokumentiert, wie die Infrastruktur konfiguriert ist.

    \item \textbf{Versionierung:}
    Die Änderungshistorie und verschiedene Versionen sind durch den Einsatz von z.B. Git immer verfügbar.

    \item \textbf{Wiederholbarkeit:}
    IaC Skripte können immer wieder für verschiedene Umgebungen ausgeführt werden.
    Dadurch lassen sich Test-, Entwicklungs- und Produktivumgebungen so ähnlich wie möglich halten.

    \item \textbf{Change Management:}
    Durch die Verwaltung in Git und die jederzeitige Nachvollziehbarkeit ist es einfacher Änderungen an der Infrastruktur strukturiert durchzuführen.

    \item \textbf{Wiederherstellungsgeschwindigkeit:}
    Ein hoher Automatisierungsgrad erlaubt die schnelle Reaktion auf Fehlerfälle in dem z.B. die letzte Version wiederhergestellt oder ein Patch veröffentlicht wird.

    \item \textbf{Zuverlässigkeit:}
    Die Vermeidung von manueller Konfiguration führt zu weniger Fehlern und höherer Reproduzierbarkeit.

    \item \textbf{Auditierung:}
    Da alle Änderungen und die gesamte Konfiguration in Git erfasst sind, können Audits auf Basis dieser Informationen einfacher durchgeführt werden.

    \item \textbf{Massenoperationen:}
    Insbesondere, wenn eine Änderung auf vielen Servern gemacht werden soll, ist der Einsatz von IaC vorteilhaft.
\end{itemize}
