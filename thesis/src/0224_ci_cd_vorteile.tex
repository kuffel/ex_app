\subsubsection{CI/CD Vorteile}\label{ci_cd_vorteile}

Zum Erreichen der erwünschten Verkürzung der Time-To-Market ist CI/CD zwingend notwendig.
Ohne per CI/CD automatisierte Prozesse lassen sich DevOps Strategien kaum erfolgreich durchsetzen.
Die Verwendung von CI/CD bietet einige Vorteile:

\begin{itemize}
    \item \textbf{Problemen frühzeitig feststellen:}
    Durch die regelmäßige Integration und automatisierte Tests werden Fehler entdeckt bevor die Ursachen weit in der Vergangenheit liegen.
    Das Problem kann besser auf eine konkrete Integration von neuem Code zurückgeführt werden.
    \footnote{Armenise, vgl.~\cite{Armenise2015}~[S.25]}

    \item \textbf{Weniger Fehler:}
    Wenn die automatisierten Tests eine hohe Abdeckung haben,
    entsteht mittel- und langfristig eine höhere Stabilität mit weniger Fehlern.
    \footnote{Abbass et. al, vgl.~\cite{Abbass2019}~[S.1]}

    \item \textbf{Schnellere Auslieferung:}
    Da die Auslieferung der Software automatisiert ist, können neue Features schneller an den Benutzer ausgeliefert werden.
    \footnote{Wiedemann, vgl.~\cite{Wiedemann2019}~[S.159]}

    \item \textbf{Bessere Reviewprozesse:}
    Bei Reviews muss der Reviewer nur Merge Requests betrachten, die zuvor durch die CI/CD Pipeline geprüft wurden.
    \footnote{Armenise, vgl.~\cite{Armenise2015}~[S.27]}
\end{itemize}
