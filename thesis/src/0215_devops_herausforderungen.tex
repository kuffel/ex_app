\subsubsection{Herausforderungen}\label{devops_herausforderungen}

Bei der Einführung von DevOps müssen sowohl technische als auch organisatorische Herausforderungen bewältigt werden.
Häufig wird Software jedoch unter Zeitdruck entwickelt und für die Verbesserung von Organisatorischen sowie technischen Problemen wird keine Zeit vorgesehen.
\footnote{Lichtenberger, vgl.~\cite{Lichtenberger2017}~[S.244]} \\

Bei zahlreichen Musterunternehmen für die DevOps Bewegung wie Netflix und Spotify wurden auf der \textsl{Grünen Wiese} gestartet und diese konnten die Themen von Anfang an angehen.
\footnote{Lichtenberger, vgl.~\cite{Lichtenberger2017}~[S.247 - S2.248]}

\paragraph{Organisatorisch}

Wenn DevOps Transformationen gestartet werden, ohne dass die Stakeholder und die Geschäftsleitung eine Notwendigkeit dafür sieht,
so scheitern ca. 50\% aller Transformationen bereits zu Beginn. \footnote{Lichtenberger, vgl.~\cite{Lichtenberger2017}~[S.246]} % `(Kotter 2012)`
Es ist notwendig einen sogenannten \textsl{Sense of Urgency} bei der Geschäftsführung und den Entscheidern zu wecken und diese
davon zu überzeugen das diese Transformation notwendig ist. \footnote{Wiedemann, vgl.~\cite{Wiedemann2019}~[S.165]}
Es darf nicht vernachlässigt werden, dass eine solche Veränderung viele Ressourcen kostet und der Wert nicht sofort ersichtlich ist. \\

Wenn das Unternehmen trotz fehlender DevOps Strategie erfolgreich ist, kann es schwierig sein diese Veränderungen zu erklären.
Unter Zeitdruck, können Konflikte alleine dadurch entstehen, dass bestimmte Aufgaben zwecks
Organisationalem Lernen auf Personen verteilt werden, welche auf den ersten Blick nicht die erfahrensten für diese Aufgaben sind.\footnote{Samulat, vgl.~\cite{Samulat2017}~[S.206]}
Diese zunächst unvermeidliche Verlangsamung muss vom Management unterstützt werden.
Eine der Grundannahmen von DevOps ist auch, dass Fehler gemacht werden dürfen, wenn auch die Konsequenzen dieser Fehler minimiert werden müssen.
Eine Kultur in der Fehler gemacht werden dürfen, hängt ganz entscheidend von der Reaktion des Managements auf diese Fehler ab. \footnote{Kim et al, vgl.~\cite{Kim2018}~[S.38]} \\

Darüber hinaus ist eine DevOps Transformation ebenfalls eine große Veränderung der bisherigen Arbeitsweisen. \footnote{Wiedemann, vgl.~\cite{Wiedemann2019}~[S.163]}
Die Angst vor Veränderung, die Kosten für Schulungen bzw. Weiterbildungen und die Verlangsamung von Prozessen zu Beginn ihrer Umstellung
ist ebenfalls zu berücksichtigen. \footnote{Kasteleiner/Schwartz, vgl.~\cite{Kasteleiner2019}~[S.211-214]}

\paragraph{Technisch}

Der technische Aspekt einer DevOps Strategie erfordert häufig eine ganze Reihe von neuen Technologien. \footnote{Puppet State of DevOps Report 2020, vgl.~\cite{PUPPET}~[S.3]}
Dafür müssen neue Technologien erlernt und eingeführt werden, Systeme umgestellt werden und vor allem Automatisierung eingeführt werden. \footnote{Samulat, vgl.~\cite{Samulat2017}~[S.205]}
Gerade die Automatisierung kann viele Unternehmen überfordern. \footnote{Samulat, vgl.~\cite{Samulat2017}~[S.211]}

Um höhere DevOps Reifegrade zu erreichen, muss eine Self Service Plattform geschaffen werden, diese lässt sich ohne
den Einsatz von SaaS Lösungen von Cloud Anbietern nicht umsetzen. \footnote{Kasteleiner/Schwartz, vgl.~\cite{Kasteleiner2019}~[S.211-214]}

Das DevOps Mantra welches von Netflix praktiziert wird lautet:

\begin{quotation}
    \textsl{You build it, you run it.}
    \footnote{König/Kugel, vgl.~\cite{Konig2019}~[S.296 - S.297]}
\end{quotation}

Um Systeme zu betreiben, müssen die Entwickler Wissen aus dem Bereich des Betriebs haben und umgekehrt.
Selbst wenn die Teams durchmischt sind mit Entwicklern und Administratoren, so kann es doch einige Zeit dauern
bis durch Knowledge Sharing das Wissen ausreichend verteilt wurde. \footnote{Puppet State of DevOps Report 2020, vgl.~\cite{PUPPET}~[S.3]}


%    \paragraph{Zusammenfassung}
%
%    - Funktionale Trennung aus Dev und Ops auflösen, interdisziplinäre Teams mit gemeinsamen Zielen und Anreizen. \footnote{Lichtenberger, vgl.~\cite{Lichtenberger2017}~[S.244]}
%    - "Culture eats Strategy for Breakfast" \footnote{Lichtenberger, vgl.~\cite{Lichtenberger2017}~[S.246]}
%    - No process definition will magi-cally deliver results without a committed, disciplined team, and organization behind it  \footnote{Ozkaya, vgl.~\cite{Ozkaya2019}~[S.4]}
%    [RESTE]
%    - "Unicorns" hatten einfache Vorraussetzungen, das die Projekte auf der Grünen Wiese gestartet sind. (IaC, Pipelines konnten von Anfang an integriert werden) `(Lichtenberger, 247-248)`
%    - DevOps ist nie das Ziel, sondern Mittel zum Zweck `(Lichtenberger, 246)`
%    - Organisation soll Lernen `(Schwarz/Kasteleiner)`
%    - Lack of Time, Standardization, Technical skill within the team `(https://puppet.com/resources/report/2020-state-of-devops-report/, )`
%    - Challenges in Adopting DevOps During `(Buhan, 8-9)`
%    - Having staff with the right technical skills
%    - Resistance to Change and Uncertainty
%    - Changing the Technology Stack and Tools
%    - Uncertainty in Responsibilities
%    - Dev will Veränderung, Ops will Stabilität: SW aus Zeitdruck häufig nicht unter Berücksichtung von Ops Problemen entwickelt \footnote{König/Kugel, vgl.~\cite{Konig2019}~[S.291]}
%    - Kosten/Risiken: Angst vor Veränderung, Schulungen und Weiterbildungen, Prozesse können während der Umstellung mehr Zeit benötigen
%    - Sharing knowledge is a key factor in the DevOps movement [BC13]. These knowledge-sharing activities are helpful because the teams learn from e
%    - Metriken sammeln um Ideen zu verifizieren \footnote{Kasteleiner/Schwartz, vgl.~\cite{Kasteleiner2019}~[S.211-214]}