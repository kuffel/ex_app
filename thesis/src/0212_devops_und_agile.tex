\subsubsection{DevOps und Agile Methoden}\label{devops_und_agile}

Die Prinzipien des agilen Manifests decken sich mit den Prinzipien von DevOps.
\footnote{Samulat, vgl.~\cite{Samulat2017}~[S.210]}
\footnote{Kim et al, vgl.~\cite{Kim2018}~[S.4 - S.5]} \\

Durch Scrum oder Kanban Boards wird die Arbeit sichtbar und transparent. \footnote{König/Kugel, vgl.~\cite{Konig2019}~[S.296]}
Darüber hinaus behandeln agile Methoden die Softwareentwicklung aus Sicht des Kunden und der Schaffung von Werten für diesen. \footnote{Mazzara, vgl.~\cite{Mazzara2019}~[S.100]}
Die Flexibilität von agilen Methoden kann durch die Erzeugung von sogenannten Minimum Viable Products (MVP) und kleinen Iterationen mit dem Wasserfallmodell nicht erreicht werden. \footnote{König/Kugel, vgl.~\cite{Konig2019}~[S.290]}

Mit agilen Methoden und DevOps lässt sich eine Produktentwicklung erreichen, die schnell und flexibel auf geänderte Anforderungen reagieren kann.
\footnote{Lichtenberger, vgl.~\cite{Lichtenberger2017}~[S.245]}
\footnote{Söllner, vgl.~\cite{Sollner2019}~[S.321]} \\

Durch den Einsatz von DevOps und agilen Methoden lässt sich mit den gleichen Ressourcen mehr erreichen und man kann unter Umständen schneller agieren als seine Mitbewerber.
Der Fokus liegt hier auf den Kunden.
Ebenso ist es hilfreich, um Verschwendung von Ressourcen zu vermeiden.
Ideen können schnell erprobt werden und es kommt seltener zur Entwicklung von nicht benötigten Features.
\footnote{Samulat, vgl.~\cite{Samulat2017}~[S.208 - S.210]}

%    [RESTE]
%    - SW Entwicklung erfordert neues Denken (Lean): `(Samulat, P. 208/210)`:
%    - Digitalisierung stellt Business und IT Strukturen in Frage: Änderungen für jeden Kulturwandel, manchmal auch Revolution
%    - Vermeidung von Verschwendung, Unregelmäßigkeit und übermäßiger Belastung
%    - T Shaped Mitarbeiter
%    - Produkt bzw. Kundenfokus
%    - Strategie kann nicht umgesetzt werden solange noch mit Taktik gekämpft wird.
%    - Commodity IT: Dienstleister stellen IT Leistungen die zu dem Preis von der internen IT nicht mehr zu erreichen sind, Grenze zwischen Kernkompetenz und Unterstützungsfunktion verschiebt sich `(Samulat, P. 209)`
%    - Lean hilft so schnell wie möglich Produkte zu erzeugen, dadurch kann mit den gleichen Ressourcen mehr erreicht wertden und man kann schneller sein als Mitbewerber.
%    - Kultur ist die langsamste und komplexeste Stellschraube im Unternehmen, "Change in Mind" erforderlich