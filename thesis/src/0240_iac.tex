\newpage
\subsection{Infrastructure as Code}\label{iac_basics}

Infrastructure as Code (IaC) bezeichnet die Verwendung von Praktiken aus der Softwareentwicklung bei der Verwaltung von Servern und anderer IT-Infrastruktur.
\footnote{Siebra et al., vgl.~\cite{Siebra2019}~[S.428]}
\footnote{Artac et al., vgl.~\cite{Artac2017}~[S.497]}

IaC soll die Verwaltung von Infrastruktur automatisierbar machen.
Sich wiederholende Aufgaben und manuelle Konfiguration sollen so minimiert werden.
\footnote{Siebra et al., vgl.~\cite{Siebra2019}~[S.428 - S.429]}

Für das DevOps \textbf{CAMS} ist IaC das A für Automatisierung.
Ohne IaC lässt sich keine Automatisierung erreichen und IaC ist somit essenziell für CI/CD und DevOps.
% `(Johann, 117)`
\footnote{Rahman, vgl.~\cite{Rahman2018c}~[S.476]} \\

Die IT-Industrie ist für eine hohe Innovationsgeschwindigkeit bekannt.
IaC soll es Entwicklern ermöglichen, Infrastruktur selbst zur Verfügung zu stellen.
Darüber hinaus wird dadurch eine höhere Entwicklungsgeschwindigkeit und schnellere Releases ermöglicht.
\footnote{Guerriero et al., vgl.~\cite{Guerriero2019}~[S.580]}
\footnote{Artac et al., vgl.~\cite{Artac2017}~[S.497]}
