\subsubsection{Zusammenfassung}\label{collaboration_vcs_zusammenfassung}

Die Grundfunktionalität jedes untersuchten VCS ist ähnlich.
Es lassen sich Remote Git Repositories erstellen und verwalten.
Alle Tools erlauben eine kostenlose Nutzung auch mit privaten Repositories.
Dabei unterscheiden sie sich je nach Tool entweder bei den verfügbaren Features
oder den Kapazitäten von CI Minuten für die Ausführung von Pipelines.

Die Weboberflächen von GitHub, GitLab und Bitbucket erlauben es, Code Reviews benutzerfreundlich durchzuführen.
Bei AWS-Code-Commit ist die Benutzererfahrung bei Merge Requests und Kommentaren deutlich eingeschränkter als bei den drei anderen VCS. \\

GitHub bietet mit GitHub Pages und Wikis die Möglichkeit, statische Websites direkt aus einem Repository heraus zu hosten.
Das kann für die Erstellung einer Dokumentationswebsite nützlich sein, da diese dann direkt zusammen mit dem Quellcode versioniert wird. \\

GitLab ist in der Starter Version bereits umfangreich nutzbar.
Das Feature von \textit{Code Approvals} wird aber erst bei der Bronze Version freigeschaltet.
Als Alleinstellungsmerkmal hat GitLab hier die Möglichkeit Tickets, Meilensteine und Kanban Boards zu verwalten. \\

BitBucket ist in der Grundversion ein reines Remote Git Repository.
Es fehlt ein Ticket/Issue System, dieses kann aber über die Anbindung von Jira realisiert werden.
Dafür fallen dann allerdings zusätzliche Kosten an. \\

Die CodeStar-Tools von AWS sind umfangreich und sollten es Unternehmen erlauben, jeden beliebigen Workflow abzubilden.
Der große Nachteil hierbei ist die komplexe Einrichtung.
Die vielfältigen Möglichkeiten sorgen dafür, dass die Benutzeroberfläche weit weniger benutzerfreundlich ist.
Mit Cloud 9 bietet AWS eine webbasierte IDE. \\

Den größten kostenlos nutzbaren Funktionsumfang für eine gehostete VCS-Lösung bietet hier GitHub.
Mit 2.000 CI Minuten bewegt sich das kostenlose Angebot in einem Bereich bei dem man bei GitLab bereits \$4 pro Benutzer bezahlen muss.
\footnote{GitLab Pricing, vgl.~\cite{GITLAB_PRICING}}