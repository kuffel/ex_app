\subsection{Kritische Reflexion}\label{kritische_reflexion}

In dieser Arbeit wurde vor allem Software verwendet, die sich bis zu einem gewissen Grad kostenlos nutzen lässt.
Gerade im Bereich DevOps und CI/CD gibt es kommerzielle Lösungen, welche versprechen alles auf einmal anzubieten.
Der Zugriff auf diese Tools ist jedoch oft nur auf Anfrage, kostenpflichtig verfügbar oder zeitlich begrenzt verfügbar.
Aus diesem Grund wurde in dieser Arbeit der Fokus auf kostenlose SaaS Angebot wie Circle CI und GitHub gelegt. \\

Die Kriterien zur Evaluierung der einzelnen Tools wurden mithilfe der Literatur, der Produktwebsites und basierend auf eigenen Versuchen ausgewählt.
Ein systematischer Vergleich war alleine aufgrund der sehr unterschiedlichen Ansätze im Bereich Kollaborationstools schwierig. \\

Die Arbeit zeigt auch, dass die Verwendung von DevOps, ein weitaus breiteres Wissen erfordert.
Durch den Einsatz von CI/CD sind wesentlich mehr Technologien, Frameworks und Tools im Einsatz.
Um DevOps verwenden zu können,s müssen Mitarbeiter also weitere Technologien beherrschen lernen. \\

Terraform verspricht einen einfachen Multi Cloud Einsatz.
Im Laufe der Arbeit zeigte sich jedoch, dass es ohne detaillierte Kenntnisse der jeweiligen Begrifflichkeiten der verschiedenen Cloud Provider,
nicht trivial ist Terraform Skript zu entwickeln.
Eine einfache Übertragung von Skripten von einem Cloud Anbieter zu einem anderen scheidet somit aus. \\
