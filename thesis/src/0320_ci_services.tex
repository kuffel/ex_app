\newpage
\subsection{CI/CD Server und Dienste}\label{ci_services}

Für CI/CD wird immer ein Server bzw.
Dienst benötigt, dessen Aufgabe die Ausführung der Jobs ist.
Diese Jobs können auf unterschiedliche Arten gestartet werden.
Üblicherweise geschieht dies durch das Erstellen eines Merge Request, den Push von neuem Code oder der Annahme eines Merge Requests. \\

Bei Stackshare werden in dieser Kategorie folgende Tools aufgeführt:
\footnote{What are the best Continuous Integration Tools?, vgl.~\cite{STACKSHARE_CI}}

\begin{itemize}
    \item \textbf{Jenkins:} \href{https://www.jenkins.io/}{https://www.jenkins.io/}
    \item \textbf{Circle CI:} \href{http://circleci.com/}{http://circleci.com/}
    \item \textbf{Travis CI:} \href{https://travis-ci.com/}{https://travis-ci.com/}
    \item \textbf{GitHub Actions:} \href{https://github.com/features/actions}{https://github.com/features/actions}
    \item \textbf{GitLab CI:} \href{https://docs.gitlab.com/runner/}{https://docs.gitlab.com/runner/}
\end{itemize}


