\paragraph{Terraform}\label{iac_tools_tools_terraform}

Terraform gehört dem Unternehmen HashiCorp.
Das Unternehmen wurde 2012 gegründet und hatte mit Vagrant eine Lösung, um das aus der Entwicklung bekannte \textsl{Works on my machine} zu beseitigen.
HashiCorp hat diverse Produkte rund um DevOps, Automatisierung, Self-Service sowie Provisionierung von Infrastruktur entwickelt und vertrieben.
\footnote{{Vagrant Founder Launches Hashicorp, vgl.~\cite{TECHCRUNCH_TERRAFORM}}}

\begin{itemize}
    \item \textbf{Vagrant:}
    Virtuelle Maschinen für Entwicklungszwecke als wiederverwendbare Konfigurationsdateien beschreiben.
    \footnote{{Development Environments Made Easy, vgl.~\cite{HASHICORP_VAGRANT}}}

    \item \textbf{Packer:}
    Erstellung von Machine Images für z.B. AWS EC2, VmWare, Azure und anderen auf Basis von Konfigurationsdateien.
    \footnote{{Build Automated Machine Images, vgl.~\cite{HASHICORP_PACKER}}}

    \item \textbf{Terraform:}
    IaC Tool mit zugehöriger CLI, Provider für verschiedenen Cloud Provider.
    \footnote{{Build Automated Machine Images, vgl.~\cite{HASHICORP_TERRAFORM}}}

    \item \textbf{Vault:}
    Server zur Verwaltung von Tokens, Passwörtern, Zertifikaten über UI, REST und CLI.
    \footnote{{Manage Secrets and Protect Sensitive Data, vgl.~\cite{HASHICORP_VAULT}}}

    \item \textbf{Consul:}
    Service Discovery, Service Meshes und Load Balancing.
    \footnote{{Service Networking Across Any Cloud, vgl.~\cite{HASHICORP_CONSUL}}}

    \item \textbf{Nomad:}
    Orchestration von Arbeitslasten über Clouds und On-Premise Installationen.
    \footnote{{Workload Orchestration Made Easy, vgl.~\cite{HASHICORP_NOMAD}}}

    \item \textbf{Boundary:}
    Zentraler Zugriff auf Infrastruktur via SSH.
    \footnote{{Simple and secure remote access, vgl.~\cite{HASHICORP_BOUNDARY}}}

    \item \textbf{Waypoint:}
    Workflows für Build, Deploy und Releases auf unterschiedlichen Cloud Angeboten.
    \footnote{{Build. Deploy. Release., vgl.~\cite{HASHICORP_WAYPOINT}}}

\end{itemize}
