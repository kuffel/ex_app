\subsubsection{Continouous Integration}\label{ci}

Die von Git und anderen VCS Systemen durchgeführte Prüfung auf Konflikte bezieht sich rein auf Konflikte der Datei- und Zeilenebene.
Im Falle von Konflikten muss ein Entwickler entscheiden, welche Zeilen übernommen oder angepasst werden müssen.
Ob die vorgenommenen Änderungen bei der Kompilierung oder im schlimmsten Falle erst zur Laufzeit zu Problemen führen, wird nicht betrachtet.

Diese Prüfung kann nur durch Kompilierung und das Ausführen von Tests vorgenommen werden.
Dabei soll CI dafür sorgen, dass diese Prüfungen regelmäßig automatisiert durchgeführt werden und nicht erst am Ende einer Entwicklung große Mengen von Änderungen zu Problemen führen.
\footnote{Abbass et. al, vgl.~\cite{Abbass2019}~[S.1]} \\

CI wird als eine der wichtigsten Praktiken in der Software Industrie angesehen und führt dazu, dass Probleme früher erkannt und behoben werden können.
Je früher ein Problem erkannt wird, desto günstiger ist dessen Behebung.
\footnote{Shweta, vgl.~\cite{Shweta2014}~[S.214]}
\footnote{Fowler, vgl.~\cite{FOWLER_CI}}


\footnote{Abbass et. al, vgl.~\cite{Abbass2019}~[S.2]}
Darüber hinaus soll CI das Problem der sogenannten \textsl{Integration Hell} vermeiden, die entstehen kann, wenn viele Änderungen erst kurz vor einem Release zusammengeführt werden müssen.
\footnote{Shweta, vgl.~\cite{Shweta2014}~[S.214]} \\

Ein typischer CI-Workflow sollte mindestens eine Kompilierung und die Ausführung von Unit Tests enthalten.
Da CI-Workflows mit Skripten definiert werden, sind die möglichen Arten von Jobs vielfältig.
Der Einsatz von CI ist die Voraussetzung für Continouous Delivery, hierdurch können automatisiert und sicher Updates ausgebracht werden.
\footnote{Shahin et al., vgl.~\cite{Shahin2017}~[S.3910 - S. 3921]}
