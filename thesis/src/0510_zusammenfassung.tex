\subsection{Zusammenfassung}\label{zusammenfassung}

Von den zuvor vorgestellten \nameref{devops_massnahmen} für eine DevOps Migration erfüllt die Implementierung vorallem die technischen Aspekte.

Durch einen hohen Grad an Automatisierung können die Teams einfach zusammenarbeiten.
Die Implementierungen können ohne großen Aufwand für ein Reviewprozess abgegeben werden, der Reviewer kann anhand der Vorschau-Umgebung
sowie des GitHub Pull-Requests Feedback geben und die Änderungen überprüfen.
Das automatisierte Testing und die regelmäßige automatische Integration sorgt dafür,
dass Änderungen schnell und sicher durchgeführt werden können. \\

Per Knopfdruck lassen sich die Änderungen anschließend ohne Ausfall, ins Produktivsystem überführen.
Im Fehlerfall kann der Merge wieder rückgängig gemacht werden und die vorherige Version wird wiederhergestellt.
Jederzeit können Entwickler dabei auch die Infrastruktur durch Anpassungen an den Terraform Skripten verändern.
Dabei ist mit der Verwendung von ECS zwar nicht alles veränderbar, es kann jedoch ohne großen Aufwand neue Infrastruktur hinzugefügt werden.
Da jede Vorschau-Umgebung vollständig vom Produktivsystem isoliert ist, besteht auch nicht die Gefahr, dass Entwicklungsarbeit das Produktivsystem gefährdet. \\

Die Implementierung erlaubt die Entwicklung von sogenannten Cloud Native Applikationen, der Betrieb eigener Infrastruktur ist nicht erforderlich.
Es kann immer auf SaaS und PaaS Dienste der AWS Cloud Infrastruktur zugegriffen werden. \\

Auf den Stufen der \nameref{devops_reifegraden} befindet sich die Lösung damit auf Stufe 4, für Stufe 5 fehlt hier lediglich die Einbindung von Sicherheitsteams.

