\subsection{Ausblick}\label{ausblick}

Die implementierte CI/CD Pipeline ist in erster Linie eine Basis für weiterführende Möglichkeiten.
Folgende Themen wurden im Rahmen der Arbeit betrachtet, jedoch nicht weiterverfolgt:

\begin{itemize}
    \item \textbf{Frontend Testing:}
    Mit dem UI-Testing Frameworks, wie \href{https://www.selenium.dev/}{Selenium}, lassen sich Abläufe in einem Webbrowser fernsteuern.
    Um einen Browser in der CI/CD Pipeline ohne Oberfläche zu starten, benötigt man einen WebDriver und einen passenden Browser.
    Circle CI enthält für gängige Browser Test Frameworks \href{https://circleci.com/docs/2.0/browser-testing/}{Dokumentation und Tutorials}.
    Durch automatisierte Frontend Test kann ein Review zusätzlich beschleunigt werden,
    da der Reviewer im besten Fall nicht manuell neue Features ausprobieren muss.

    \item \textbf{IaC Linting:}
    Im Rahmen der CI/CD Pipeline werden die Terraform Skripte mit dem Kommando \texttt{terraform validate} überprüft.
    Dabei wird jedoch nur die Syntax geprüft, eine Prüfung der Parameter wird dabei nicht durchgeführt.
    Ein Fehler in diesen Parametern würde somit erst beim Deployment auffallen.
    Durch das Preview Deployment werden diese Probleme früh erkannt.
    Um diese Probleme noch früher zu erkennen, könnte mit der Anwendung \href{https://github.com/terraform-linters/tflint}{tflint} eine Prüfung der Parameter vorgenommen werden.

    \item \textbf{Vollständig Automatisierung:}
    Mit vollständiger Test-Abdeckung wäre es auch möglich, eine vollständige automatisches CI/CD-Pipeline zu erstellen.
    Der Merge Request würde regulär durchgeführt werden.
    Es würde nach Abschluss aller Tests und Prüfungen ein automatischer Merge stattfinden.
    Ohne die Kontrolle durch einen menschlichen Reviewer müssten dafür eine höhere Test-Abdeckung, sowie die Möglichkeit von automatischen Rollbacks im Fehlerfall geschaffen werden.

    \item \textbf{DevSecOps:}
    Um die Stufe 5 der DevOps Reifegrade zu erreichen ist eine Einbindung von Sicherheitsteams in die Prozesse erforderlich.
    Diese Einbindung erfolgt oft unter dem Begriff DevSecOps.
    Hierfür könnten Prüfungen bereits in die CI/CD Pipeline eingebaut werden.
    Mit den Tools \href{https://github.com/tfsec/tfsec}{tfsec} und \href{https://kics.io/}{kics} lassen sich Terraform Skripte statisch auf Sicherheitsprobleme hin überprüfen.
    Für das verwendete Phoenix Framework gibt es mit \href{https://github.com/nccgroup/sobelow}{sobelow} ein weiteres statisches Codeanalyse Tool,
    welches automatisch Prüfungen auf Sicherheitsprobleme durchführen kann.
\end{itemize}

% Circle CI Paid Plans mit Docker Layer Caching um Build Zeiten zu senken
% Eigene Metriken und Telemetrie
% A/B Testing für Features
% Automatische Lasttests via z.B. loadimpact.io
%-Serverless CI/CD mit AWS Lambda
%-AWS Cloud9 ist ein intressanter Weg (Pair Programming, No Software Install)
%-AWS CodeGuru
%-User Feedback
% IaC hat oft sehr unterschiedliche Ausprägungen
% Cloud Provider haben alle Kubernetes als Option
% AWS geht mit ECS einen eigenen Weg
% AWS soweit vorraus das Azure/Google eigene Guides für ihre Eigene Terminologie vs AWS haben
% AWS CodeStar Tools für große Unternehmen die alles auf AWS Integrieren möchten gut.

