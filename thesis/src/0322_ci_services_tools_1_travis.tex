\paragraph{Travis CI}\label{ci_services_tools_travis}

Travic CI wurde 2011 von einem Berliner Unternehmen als gehosteter Build Service gestartet. \footnote{Crunchbase Travic CI , vgl.~\cite{CRUNCHBASE_TRAVIS_CI}}
Es findet bei über 900.000 Projekten Verwendung und ermöglicht das kostenlose Bauen von Open-Source-Projekte. \footnote{Travis Homepage , vgl.~\cite{TRAVIS_HOMEPAGE}} \\

Travis CI steht als SaaS zur Verfügung.
Die Kollaborationstools GitHub, GitLab, Bitbucket und Assembla werden unterstützt.
Die Builds laufen immer in einem Docker Container, somit wird von Travis CI jede Programmiersprache - für die es Docker Images gibt - unterstützt.
Die Erstellung der Build Konfiguration erfolgt über eine YAML Datei und eine DSL.
Travis CI bietet Agents für die Ausführung von Builds auf Windows, Linux und MacOS. \\

Durch die Verfügbarkeit von Templates und einer im Vergleich zu Jenkins kleineren Dokumentation lassen sich
Builds schnell einrichten. \\

In der kostenlosen Version sind 10.000 Credits enthalten.
Diese werden mit jeder Minuten, die ein Build läuft, reduziert.
Es erfolgt hierbei eine Abrechnung je nach Betriebssystem des Agents.
Linux wird mit 10 Credits/Minute, Windows mit 20 Credits/Minute und MacOS mit 50 Credits/Minute abgerechnet. \footnote{Travis CI Billing, vgl.~\cite{TRAVIS_BILLING}} \\

In der kostenlosen Version kann nur ein Build parallel gestartet werden.
Sind die Credits verbraucht können ab \$69 pro Monat unbegrenzte zusätzliche Credits erworben werden.
Es lässt sich hierbei jedoch trotzdem nur ein paralleler Build durchführen. \footnote{Travis Plans, vgl.~\cite{TRAVIS_PLANS}} \\

Für \$249 pro Monat lassen sich 5 parallele Builds durchführen.
Benötigt man mehr so muss ein Travis Server selbst gehostet werden.
Der Preis hierfür liegt bei \$8.000 für 20 Benutzer im Jahr. \footnote{Travis CI Enterprise, vgl.~\cite{TRAVIS_ENTERPRISE}}




