\section{Einleitung}\label{einleitung}

Die meisten Unternehmen sind auf eine funktionierende IT angewiesen, selbst wenn diese nicht deren Produkt ist,
sondern als interne Dienstleistung verwendet wird. \footnote{Alt, Auth, Kögler, vgl.~\cite{Alt2017}~[S.217- S.220]} \footnote{Wiedemann, vgl.~\cite{Wiedemann2019}~[S.157]}
Ein Ausfall von IT-Systemen führt i.d.R. direkt oder indirekt zu finanziellen Einbußen.
Wenn die System nur intern verwendet werden, kommt es zunächst zu einem Produktivitätsverlust.
Für Unternehmen deren Produktpalette hauptsächlich softwaregestützte Dienstleistungen sind, entstehen jedoch direkt Umsatzeinbußen. \\

Der reibungslose Betrieb von IT-Systemen ist somit geschäftskritisch. \footnote{Alt, Auth, Kögler, vgl.~\cite{Alt2017}~[S.217- S.220]}
Darüber hinaus stellt die deutlich höhere Innovationsgeschwindigkeit von IT-Systemen enorme Anforderungen an IT-Abteilungen.
Diese müssen immer schneller auf neue Gegebenheiten und Kundenanforderungen reagieren. \\

Insbesondere Unternehmen, deren Hauptprodukt Software ist, haben somit ein gesteigertes finanzielles Interesse an dem reibungslosem Betrieb der IT trotz häufiger Änderungen.
Bekannte Beispiele wie Netflix, Spotify und PayPal haben Dinge aus der \emph{realen} Welt wie Filmverleih, Tonträger und Bargeld in Software umgesetzt.
Unternehmen wie Airbnb oder Uber haben auch dafür gesorgt das die Trennung zwischen der \emph{realen} Welt und IT nicht mehr so einfach differenziert werden kann.
\footnote{Mazzara, vgl.~\cite{Mazzara2019}~[S.100]} \\

Gerade diese Unternehmen haben ihre Branchen verändert und mitunter sogar andere Konkurrenten verdrängt.
Sie haben IT-Innovationen nicht verpasst, sondern es geschafft Treiber dieser Innovationen zu werden.
Wenn Unternehmen längere Zeit IT-Innovationen verpassen, kann ihre Wettbewerbsfähigkeit dadurch gefährdet sein. \footnote{Alt, Auth, Kögler, vgl.~\cite{Alt2017}~[S.217- S.220]} \\

Denn durch die Entwicklungen im Bereich von mobilen Anwendungen, die im Prinzip von jedem veröffentlicht werden können,
laufen Unternehmen Gefahr von einem wesentlich kleineren und schnelleren neuen Anbieter bzw. Startups verdrängt zu werden.
Dabei gewinnt oft nicht derjenige, der die größte Qualität liefert, sondern derjenige der neue Anforderungen schneller umsetzen kann. \\

Um Innovationen umsetzten zu können, müssen Unternehmen ihre IT-Systeme anpassen und verändern können, ohne den Betrieb dabei zu gefährden.
Gerade bei der Entwicklung von Individualsoftware hat sich gezeigt, dass agile Methoden den klassischen Modellen wie z.B. dem Wasserfallmodell überlegen sind.
Dies ist zumindest der Fall, wenn die Anforderungen zu Beginn nicht klar definiert sind oder sich im Laufe der Entwicklung verändern. \\

Der Begriff, der sich im Rahmen von agiler Softwareentwicklung häufig neben Scrum Verwendung findet, lautet DevOps.
Dabei handelt es sich um ein Kunstwort aus den Begriffen Development und Operations somit stellt es keine klar definierte Methode oder Technik. \footnote{Alt, Auth, Kögler, vgl.~\cite{Alt2017}~[S.218]}
So ist DevOps bspw. das Resultat der Anwendung von Prinzipien aus der Herstellung von physischen Gütern bei Toyota \footnote{Kim et. al, vgl.~\cite{Kim2018}~[S.3 - S.4]} \\

Unternehmen wie Facebook, Netflix und Amazon haben hierbei den Trend zu DevOps beschleunigt.
Bereits 2013 hat Facebook wöchentlich neue Versionen ihrer Anwendung direkt in ihren Produktivsystemen installiert und somit zum Kunden gebracht. \footnote{Feitelson, Beck, vgl.~\cite{Feitelson2013}~[S.13]} \\

Eine Studie aus dem Jahr 2019 brachte hervor, dass Unternehmen, die als sogenannte DevOps Leader gelten, im Durchschnitt 208 mal häufiger Software in ihren Produktivsystemen bereitstellen als ihre Mitbewerber.
Der Maximalwert in dieser Umfrage lag hier bei 1460 Deployments im Jahr und der niedrigste Wert bei 7 Deployments im Jahr.
\footnote{https://www.zdnet.com/article/devops-leaders-deliver-software-200-times-more-frequently-than-their-peers-study-shows/, vgl.~\cite{ZDNET_DEVOPS}} \\

Netflix gibt an, tausende Male am Tag Änderungen in die Produktion zu überführen.
Amazon hat während ihrer Umstellung von eigenen Servern auf AWS durchschnittlich alle 11.7 Sekunden neuen Code bereitgestellt.
\footnote{https://techbeacon.com/devops/10-companies-killing-it-devops, vgl.~\cite{TECHBEACON}} \\

Die vorgenannten Beispiele haben viele Unternehmen dazu veranlasst, eigene Strategien im Bereich DevOps und Continuous Delivery zu verfolgen.