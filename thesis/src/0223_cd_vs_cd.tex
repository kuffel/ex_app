\subsubsection{Continouous Deployment vs. Continouous Delivery}\label{cd_vs_cd}

Die Begriffe Continouous Deployment und Continouous Delivery werden schon allein aufgrund ihrer gleichlautenden Akronyme häufig gleich verwendet.
Der Begriff Delivery beschreibt laut Literatur eher die Auslieferung an den Endkunden, welcher ein Deployment vorausgeht. \footnote{Abbass et. al, vgl.~\cite{Abbass2019}~[S.1]}

Das Ziel von CI/CD ist die Reduzierung von manuellen Tätigkeiten, die Erhöhung der Qualität und die Beschleunigung des Releasezyklus.
\footnote{Mazzara, vgl.~\cite{Mazzara2019}~[S.103 - 104]}

Um Continouous Delivery zu ermöglichen ist es notwendig, dass die CI-Pipeline eine hohe Qualität sicherstellt.
Dazu muss es Prüfungen auf schlechte Commits, zu lange Build Zeiten und Test Abdeckung in der Pipeline geben.
\footnote{Abbass et. al, vgl.~\cite{Abbass2019}~[S.3 - S.4]}

