\paragraph{Circle CI}\label{ci_services_tools_circle}

Circle CI wurde 2011 in San Francisco gegründet und steht wie Travis CI als SaaS zur Verfügung.
Für den Code können bei Circle CI nur GitHub und Bitbucket als Quelle verwendet werden. \footnote{Crunchbase Circle CI , vgl.~\cite{CRUNCHBASE_CIRCLE_CI}}

Bei einem Build kann zwischen einem machine-executor und einem docker-executor ausgewählt werden.
Dabei werden die Docker Images  von Circle CI zur Verfügung gestellt.
Es lassen sich aber andere Docker Images verwenden.
Circle CI unterstützt die Ausführung von Builds auf Windows, Linux und MacOS.
Es kann ebenfalls die Größe der Instanzen bestimmt werden. \footnote{Choosing an Executor Type , vgl.~\cite{CIRCLE_EXECUTORS}} \\

Auch bei Circle CI lassen sich Builds einfach mit einem Assistenen einrichten.
Die Definition erfolgt hier ebenfalls über eine YAML Datei und einer DSL. \footnote{Configuration Introduction , vgl.~\cite{CIRCLE_CONFIG}} \\

Eine der Besonderheiten von Circle CI sind die sogenannten Orbs.
Hierbei handelt es sich um vorgefertigte Build Schritte, die über eine Repository von Circle CI und der Community zur Verfügung gestellt werden.
Mit Orbs lassen sich die Installation von CLI Tools mit einem Befehl in dem jeweiligen Executor einrichten.
Zusätzlich bietet Circle auch ein Cache im Dateisystem an.
Es können die Artefakte eines Build Schritts gespeichert und wiederverwendet werden. \footnote{Orbs Introduction , vgl.~\cite{CIRCLE_ORBS}} \\

In der kostenlosen Variante stehen nur Linux und Windows mit zusammen 2.500 Credits pro Woche zur Verfügung.
Die Kosten für 3 Benuzuer und 25.000 Credits pro Woche belaufen sich auf \$30 pro Monat.
Für \$15 zusätzlich pro Monat können weitere 25.000 Credits erworben werden.
Jeder weitere Benutzer kostet zusätzlich \$15 pro Monat.

Die Credits werden bei Circle CI nach Größe und Betriebssystem der Build Runners abgerechnet.
Die Spanne variiert dabei von 5 Credits/Minute für Builds auf einer kleinen Linux Instanz
bis hin zu 500 Credits/Minute auf einer NVIDIA Tesla GPU Instanz.

In der kostenpflichtigen Variante steht darüberhinaus Docker Layer Caching zur Verfügung.
Hierbei werden bei den Builds nur noch die Änderungen in den jeweiligen Images gebaut, das führt zu einer deutlichen Senkung der Build Zeiten.
Wird Docker Layer Caching verwendet, so werden pauschal 200 Credits für jeden Build abgezogen.

In der Scale Variante besteht zusätzlich die Möglichkeit, die Build Infrastruktur auf eigenen Servern zu betreiben, 24/7 Support,
GPU und größere Instanztypen zu verwenden.
Der Preis hierfür ist jedoch nur auf Anfrage erhältlich. \footnote{Circle CI Pricing , vgl.~\cite{CIRCLE_PRICING}} \\

