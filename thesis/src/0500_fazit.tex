\section{Fazit}\label{fazit}

Eine Umsetzung einer DevOps Strategie ist bei Neuentwicklungen einfacher, da es nicht notwendig ist eingefahrene Abläufe zu ändern.
Die technischen Voraussetzungen lassen sich bei neuen Projekten ebenfalls von Anfang an berücksichtigen.
Eine nachträgliche DevOps Migration lässt sich mit den beschriebenen Wegen jedoch schrittweise ebenfalls durchführen. \\

Die Verwendung von agilen Methoden und DevOps erlaubt es, dass Unternehmen kleine Iterationen durchzuführen, kontinuierliches Feedback und eine schnelle Reaktion auf geänderte Anforderungen.
Ein Einsatz von agilen Methoden ohne kontinuierliches Deployment der Änderungen macht kaum Sinn, da DevOps ein Enabler für agile Entwicklung ist.
DevOps und agile Methoden sollten also immer gemeinsam verwendet werden. \\

Bei der Betrachtung der Kollaborationstools fällt auf, dass sich Git als Versionskontrolle durchgesetzt hat.
Andere Systeme werden kaum verwendet.
Die Anbieter von Webbasierten Tools wie GitLab, GitHub und BitBucket bieten alle einen ähnlichen Funktionsumfang.
GitLab tritt dabei als Lösung hervor, in der alles in einem Tool erledigt werden kann.
Neben der Versionskontrolle ist dort ein Ticketsystem, CI/CD Pipelines und Tools fürs Projektmanagement sowie Sicherheit/Compliance vorhanden. \\

Bei den CI/CD Serverlösungen wird oft Jenkins verwendet.
Dafür ist es jedoch notwendig diesen auf eigenen Servern zu installieren.
Mit über 1.500 Plugins kann jedes Szenario mit Jenkins abgebildet werden.
Das macht die Verwaltung und insbesondere die Durchführung von Updates komplex.
Bei den cloudbasierten CI/CD Systemen kommt in der Regel YAML mit einer DSL zum Einsatz.
Die Skripte sind deutlich lesbarer als die Groovy basierten Jenkinsfiles. \\

Mit Circle CI hat man eine gute Integration in GitHub.
Es lässt sich allerdings auch nur mit BitBucket und GitHub verwenden.
Besonders die Orbs und das Docker Layer Caching erlauben hier die einfache Erstellung schneller CI/CD Pipelines. \\

Für die Provisionierung von Infrastruktur in einer Public Cloud erscheint Terraform die erste Wahl zu sein.
Keines der betrachteten Tools stellt die Provisionierung von Infrastruktur an erste Stelle.
Die anderen Tools sind eher zur Verwaltung bereits vorhandener Infrastruktur geeignet.
Durch die Unterstützung von sogenannten Providern lassen sich mit Terraform auch bereits vorhandene Dienste verwalten.
Die Abgrenzung zwischen Provisionierung und Konfigurationsmanagement ist somit auch bei Terraform nicht ganz deutlich. \\

Die Dienste von AWS sind im Vergleich zu Google Cloud und Azure wesentlich umfangreicher.
Gerade im Bereich CaaS bietet AWS mit ECS eine Alternative zu Kubernetes an.
Die Einarbeitung in Kubernetes ist deutlich aufwendiger als in ECS und mit Fargate bietet Amazon die Möglichkeit Container serverlos zu betreiben.
Bei der Verwendung von ECS zahlt man ausschließlich für die Ressourcen anders als bei den meisten Kubernetes Lösungen, bei denen bereits das Dashboard kostenpflichtig ist. \\

Für die technische Umsetzung von CI/CD Pipelines gibt eine Vielzahl von Möglichkeiten.
Die in dieser Arbeit implementierte Pipeline könnte in ähnlicher Form auch mit anderen Tools umgesetzt werden.
Die vorgeschlagene Lösung lässt sich jedoch vollkommen kostenlos beginnen.
Lediglich die Kosten für die Infrastruktur auf AWS sind zu entrichten.



