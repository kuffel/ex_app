\newpage
\subsubsection{Vergleich}\label{ci_services_vergleich}

% Please add the following required packages to your document preamble:
% \usepackage{multirow}
% \usepackage{graphicx}
\begin{table}[H]
\centering
\resizebox{\textwidth}{!}{%
\begin{tabular}{r|c|c|c|c|c|}
\cline{2-6}
\multicolumn{1}{l|}{} & \textbf{Jenkins} & \textbf{Travis CI} & \textbf{Circle CI} & \textbf{GitHub Actions} & \textbf{GitLab CI} \\ \hline
\multicolumn{1}{|r|}{\textbf{Unterstützte Kollaborationstools}} & \begin{tabular}[c]{@{}c@{}}Alle \\ (Über Plugins)\end{tabular} & \begin{tabular}[c]{@{}c@{}}GitHub, GitLab, \\ Bitbucket, Assembla\end{tabular} & \begin{tabular}[c]{@{}c@{}}GitHub\\ Bitbucket\end{tabular} & GitHub & GitLab \\ \hline
\multicolumn{1}{|r|}{\textbf{Authentifizierung}} & LDAP, Email & \begin{tabular}[c]{@{}c@{}}Github, Bitbucket, \\ Gitlab, Assembla\end{tabular} & \begin{tabular}[c]{@{}c@{}}Github, \\ Bitbucket\end{tabular} & siehe GitHub & LDAP, Email \\ \hline
\multicolumn{1}{|r|}{\textbf{Pipelines as Code}} & Groovy Scripte & YAML & YAML & YAML & YAML \\ \hline
\multicolumn{1}{|r|}{\textbf{Verwaltung von Secrets}} & Über Plugins & \begin{tabular}[c]{@{}c@{}}Über Environment\\ Variablen\end{tabular} & \begin{tabular}[c]{@{}c@{}}Über Environment\\ Variablen\end{tabular} & \begin{tabular}[c]{@{}c@{}}Über Environment\\ Variablen\end{tabular} & \begin{tabular}[c]{@{}c@{}}Über Environment\\ Variablen\end{tabular} \\ \hline
\multicolumn{1}{|r|}{\textbf{Unterstützte Programmiersprachen}} & \begin{tabular}[c]{@{}c@{}}Abhängig vom\\ Agent\end{tabular} & \begin{tabular}[c]{@{}c@{}}Alle die in \\ Docker ausgeführt\\ werden können\end{tabular} & \begin{tabular}[c]{@{}c@{}}Alle die in \\ Docker ausgeführt\\ werden können\end{tabular} & \begin{tabular}[c]{@{}c@{}}Alle die in \\ Docker ausgeführt\\ werden können\end{tabular} & \begin{tabular}[c]{@{}c@{}}Alle die in \\ Docker ausgeführt\\ werden können\end{tabular} \\ \hline
\multicolumn{1}{|r|}{\textbf{Unterstützte Betriebssysteme}} & \begin{tabular}[c]{@{}c@{}}Linux, \\ Windows, \\ MacOS\end{tabular} & \begin{tabular}[c]{@{}c@{}}Linux, \\ Windows, \\ MacOS\end{tabular} & \begin{tabular}[c]{@{}c@{}}Linux, \\ Windows, \\ MacOS\end{tabular} & \begin{tabular}[c]{@{}c@{}}Linux, \\ Windows, \\ MacOS\end{tabular} & \begin{tabular}[c]{@{}c@{}}Linux, \\ Windows, \\ MacOS\end{tabular} \\ \hline
\multicolumn{1}{|r|}{\textbf{Container support}} & \begin{tabular}[c]{@{}c@{}}Ja \\ (Über Plugins)\end{tabular} & Ja & Ja & Ja & Ja \\ \hline
\multicolumn{1}{|r|}{\textbf{Usability}} & \begin{tabular}[c]{@{}c@{}}Komplexe\\ Einrichtung\end{tabular} & \begin{tabular}[c]{@{}c@{}}einfache Tutorials\\ und Templates\end{tabular} & \begin{tabular}[c]{@{}c@{}}einfache Tutorials\\ und Templates\end{tabular} & \begin{tabular}[c]{@{}c@{}}einfache Tutorials\\ und Templates\end{tabular} & \begin{tabular}[c]{@{}c@{}}einfache Tutorials\\ und Templates\end{tabular} \\ \hline
\multicolumn{1}{|r|}{\textbf{Self Hosting}} & Ja & \begin{tabular}[c]{@{}c@{}}Ja\\ (für viele Bulds\\ bevorzugt)\end{tabular} & Ja & Ja & Ja \\ \hline
\multicolumn{1}{|r|}{\multirow{3}{*}{\textbf{Kosten}}} & \begin{tabular}[c]{@{}c@{}}On Premise\\ (Infrastruktur \\ und\\ Wartungskosten)\end{tabular} & \begin{tabular}[c]{@{}c@{}}Free\\ 10.000 Credits\\ Private and OpenSource\\ Unlimited Users\end{tabular} & \begin{tabular}[c]{@{}c@{}}Free\\ Linux/Windows\\ 2.500 Credits/Woche\end{tabular} & siehe GitHub & siehe GitLab \\ \cline{2-6}
\multicolumn{1}{|r|}{} & \multicolumn{1}{l|}{} & \begin{tabular}[c]{@{}c@{}}Concurrent Job Plans\\ 1 Job \$69/Monat\\ 2 Job \$129/Monat\\ 5 Job \$249/Monat\end{tabular} & \begin{tabular}[c]{@{}c@{}}Performance\\ \$30/Monat\\ MacOS\\ 25.000 Credits/Woche\\ 3 Benutzer\\ 80 Concurrent Jobs\\ Docker Layer Caching\end{tabular} &  &  \\ \cline{2-6}
\multicolumn{1}{|r|}{} & \multicolumn{1}{l|}{} & \begin{tabular}[c]{@{}c@{}}Unlimited\\ On Premise\\ \$8000 pro 20 User/Jahr\end{tabular} & \begin{tabular}[c]{@{}c@{}}Scale\\ GPU Instances\end{tabular} &  &  \\ \hline
\multicolumn{1}{|r|}{\textbf{Rang auf Stackshare}} & 1. (33.00 Stacks) & 2. (6.800 Stacks) & 3. (7350 Stacks) & - & 4. (1.590 Stacks) \\ \hline
\multicolumn{1}{|r|}{\textbf{Besonderheiten}} & Viele Plugins & Einfache Einrichtung & \begin{tabular}[c]{@{}c@{}}Einfache Einrichtung\\ Integration in GitHub\\ Viele Templates\\ Orbs\end{tabular} & \begin{tabular}[c]{@{}c@{}}Einfache Einrichtung\\ Integration in GitHub\\ Viele Templates\end{tabular} & \begin{tabular}[c]{@{}c@{}}Einfache Einrichtung\\ Integration in GitLab\end{tabular} \\ \hline
\end{tabular}%
}
\caption{Vergleich CI/CD Tools}
\label{tab:ci_cd_tools_vergleich}
\end{table}


%- Preise
%    - Jenkins
%        - Kein gehosteter Dienst, Infrastruktur Kosten
%    - Travis CI
%        - Free Plan
%            - 10.000 Credits (Credits per build minute depend on os: Linux 10, Windows 20, MacOS 50)
%            - Unlimited unqique users
%            - Private \& Open-Source Repos
%            - Windows, Linux Mac OS
%        - 1 concurrent job Plan (\$69/month)
%        - 2 concurrent job Plan (\$129/month)
%        - 5 concurrent job Plan (\$249/month)
%        - Unlimited concurrent builds => Travis CI Enterprise (On Premise, \$8000, per 20 Users/Year)
%    - Circle CI `(https://circleci.com/pricing/)`
%        - Free (\$0, 2500 Free Credits/Week)
%            - Docker Linux
%            - Linux VM
%            - Windows VM
%            - 1 Concurrent Job
%        - Performance (\$15/month for first 3 users (\$15/month for additional users))
%            - Starts at 25,000 credits for \$15
%            - Free + macOS VM
%            - Docker Layer Caching
%            - 80 Concurrent jobs
%            - Ticket based Global Support (8x5)
%        - Scale (Custom pricing)
%            - GPU-Nvidia
%            - Windows GPU-Nvidia
%            - 24x5 and 24x7 available
%        - About the credits:
%            - 200 per Job with Docker Layer Caching
%            - Depending on VM OS, CPU, RAM => 5 Credits/Min (Small) to 500 for Nvidia Tesla T4
%    - GitHub Actions `(https://github.com/pricing)`
%        - Included in GitHub Pricing Tiers
%    - GitLab CI `(https://about.gitlab.com/pricing/)`
%        - Included in GitLab Pricing Tiers
%    - AWS CodePipeline `(https://aws.amazon.com/de/codepipeline/pricing/)`
%        - 1\$ per Pipeline per Month + (Additional AWS Ressources)
%    - CodeShip IO `(https://codeship.com/pricing)`
%        - Free 100 Builds per Month, Unlimited Projects, Unlimited Team Members
%        - Starter: 1 build at a time, 2 test pipelines  (\$49 / month)
%        - Essential: 2 build at a time, 2 test pipelines  (\$99 / month)
%        - Plus: 3 build at a time, 3 test pipelines  (\$199 / month)
%        - Power: 4 build at a time, 4 test pipelines  (\$399 / month)
%        - Premium: 6 build at a time, 6 test pipelines  (\$999 / month)
%    - Azure DevOps `(https://azure.microsoft.com/en-us/pricing/details/devops/azure-devops-services/)`
%        - Free:
%            - 1 Free Microsoft hosted Repo and 1 Free Self Hosted CI/CD, 1800 Build Minutes per Month, 1 parallel job
%            - First 5 Users free, then \$ 6 per Month per User
%            - First 2 GB free
%        - Pay-as-you-go:
%            - \$15 for additional paralled jobs with unlimited minutes
%            - \$6 User / Month
%            - \$1 per Extra GB
%- Commercial Support
%    - Jenkins: Kein eigenes Unternehmen, Support nur über Dritte.
%    - Travis CI: Verfügbar als Teil der regulären Pläne
%    - Circle CI: Verfügbar als Teil der regulären Pläne
%    - GitHub Actions: : Verfügbar als Teil der regulären Pläne
%    - GitLab CI: Verfügbar als Teil der regulären Pläne
%    - AWS CodePipeline: Through AWS support \$ 29 or 3% of monthly bill
%    - CodeShip IO: Verfügbar als Teil der regulären Pläne
%    - Azure DevOps: Through Azure Support, starting at \$29 `(https://azure.microsoft.com/en-us/support/plans/)`
%- Unterstützte Kollaborationstools
%    - Jenkins: Durch Plugins wirklich alle die es gibt, Code wird einfach per HTTPS/SSH geklont
%    - Travis CI: GitHub, GitLab, Bitbucket, Assembla `(https://docs.travis-ci.com/user/tutorial/)`
%    - Circle CI:  GitHub, Bitbucket
%    - GitHub Actions: GitHub
%    - GitLab CI: GitLab
%    - AWS CodePipeline: AWS Code Commit, GitHub, Bitbucket, AWS S3, AWS ECR
%    - CodeShip IO: Github, Bitbucket, GitLab or other self hosted versions of them
%    - Azure DevOps: Azure Code, GitHub, Bitbucket
%- Authentifizierung
%    - Jenkins: LDAP, Username, Password
%    - Travis CI: Github, Bitbucket, Gitlab, Assembla
%    - Circle CI: Github, Bitbucket
%    - GitHub Actions: Same as Github
%    - GitLab CI: Same as Github
%    - AWS CodePipeline: AWS IAM
%    - CodeShip IO: E-Mail, GitHub, GitLab, BitBucket
%    - Azure DevOps: Microsoft AD, Github
%- Rollen und Rechtesystem
%    - Jenkins: Auf Job Ebene über Gruppen.
%    - Travis CI: No permissions on Job Level
%    - Circle CI: No permissions on Job Level
%    - GitHub Actions: No permissions on Job Level
%    - GitLab CI: Same as for the project, but not more
%    - AWS CodePipeline: Through IAM
%    - CodeShip IO: No permissions on Job Level
%    - Azure DevOps: Granular permissions with groups
%- Pipelines as Code
%    - Jenkins: Declarative Pipeline in Groovy Code
%    - Travis CI: YAML Files
%    - Circle CI: YAML Files
%    - GitHub Actions: YAML Files
%    - GitLab CI: YAML Files
%    - AWS CodePipeline: YAML Files
%    - CodeShip IO: No, setup via UI contains code snippets but no single file
%    - Azure DevOps: YAML Files
%- Verwaltung von Secrets
%    - Jenkins: Über Plugins lassen sich Secrets in die Jobs inizieren.
%    - Travis CI: https://docs.travis-ci.com/user/best-practices-security/
%    - Circle CI: https://circleci.com/blog/keep-environment-variables-private-with-secret-masking/
%    - GitHub Actions: https://docs.github.com/en/free-pro-team@latest/actions/reference/encrypted-secrets
%    - GitLab CI: https://docs.gitlab.com/ee/ci/variables/
%    - AWS CodePipeline: Secrets Manager https://aws.amazon.com/de/about-aws/whats-new/2019/11/aws-codebuild-adds-support-for-aws-secrets-manager/
%    - CodeShip IO: https://docs.cloudbees.com/docs/cloudbees-codeship/latest/basic-builds-and-configuration/set-environment-variables
%    - Azure DevOps: Azure Key Vault https://azuredevopslabs.com/labs/vstsextend/azurekeyvault/
%- Unterstützte Programmiersprachen
%    - Jenkins: Alles was der Agent ausführen kann, Bash Skripte die mit 0 Ende um einen Erfolg zu signalisieren.
%    - Travis CI: `https://docs.travis-ci.com/user/languages/`
%    - Circle CI: `https://circleci.com/docs/2.0/demo-apps/`
%    - GitHub Actions: Every language that has a docker image available
%    - GitLab CI: Every language that has a docker image available
%    - AWS CodePipeline: Every language that has a docker image available
%    - CodeShip IO: Ruby, Python, Java (Groovy, Scala, Clojure), Go, Javascript, PHP, Rust, Dart, Elixir or custom commands
%    - Azure DevOps: Every language that has a docker image available
%- Unterstützte Betriebssysteme
%    - Jenkins: Alle auf denen der Jenkins Agent installiert oder die per SSH angesporchen werden können.
%    - Travis CI: Linux, Windows, MacOS
%    - Circle CI: Linux, Windows, MacOS
%    - GitHub Actions: Linux, Windows, MacOS
%    - GitLab CI: Linux, Windows, MacOS
%    - AWS CodePipeline: All that EC2 can run Linux, Windows, MacOS
%    - CodeShip IO: `(https://docs.cloudbees.com/docs/cloudbees-codeship/latest/pro-jet-cli/installation)`
%    - Azure DevOps: `(https://docs.microsoft.com/en-us/azure/devops/pipelines/agents/v2-osx?view=azure-devops)`
%- Container support
%    - Jenkins: Yes, through plugins
%    - Travis CI: Yes
%    - Circle CI: Yes
%    - GitHub Actions: Yes
%    - GitLab CI: Yes
%    - AWS CodePipeline: Yes
%    - CodeShip IO: Only in Pro Version
%    - Azure DevOps: Yes
%- Dokumentation
%    - Jenkins: Sehr umfangreich
%    - Travis CI: Umfangreich, Tutorial beim anglegen des ersten Projekts reicht aus.
%    - Circle CI: Umfangreich, Tutorial beim anglegen des ersten Projekts reicht aus.
%    - GitHub Actions: Umfangreich,
%    - GitLab CI: Umfangreich
%    - AWS CodePipeline: Zu Umfangreich
%    - CodeShip IO: Ja aber unterteil in Pro und Basic, das macht es verwirrend
%    - Azure DevOps: Zu Umfangreich
%- Usability
%    - Jenkins: Ohne Erfahrung oder Dokumentation geht hier nix.
%    - Travis CI: Tutorial beim anglegen des ersten Projekts reicht aus.
%    - Circle CI: Tutorial beim anglegen des ersten Projekts reicht aus.
%    - GitHub Actions: Tutorial beim anglegen des ersten Projekts reicht aus.
%    - GitLab CI: Tutorial reicht nicht ganz
%    - AWS CodePipeline: Kein Tutorial, sehr viele Einstellungen möglich
%    - CodeShip IO: Tutorial, komplexe Einstellungen machen es aber nicht einfach.
%    - Azure DevOps: Viel zu viele Möglichkeiten, ohne Dokumentation nicht einzurichten
%- Self Hosting
%    - Jenkins: Ja
%    - Travis CI: Yes, for more build this seems to be the favourite Option
%    - Circle CI: Yes
%    - GitHub Actions: Yes via GitHub Enterprise
%    - GitLab CI: Yes
%    - AWS CodePipeline: Nein
%    - CodeShip IO: Nein
%    - Azure DevOps: Ja Azure DevOps Server
%- Besondere Features
%    - Jenkins: Viele Plugins erlauben sehr individuelle Konfiguration.
%    - Travis CI: Sehr einfach einzurichten
%    - Circle CI: Sehr einfach einzurichten, Orbs
%    - GitHub Actions: Sehr einfach einzurichten
%    - GitLab CI: Sehr einfach einzurichten
%    - AWS CodePipeline: Volles AWS Ökosystem nutzbar
%    - CodeShip IO: Einfache Einrichtung wenn die Programmiersprache unterstützt wird
%    - Azure DevOps: Scrum, Agile usw einstellbar. Enthält Boards, Sprint Plannung, DevOps, CI/CD, VCS in einer Oberfläche
%- Rang auf Stackshare
%    - Jenkins: 1. 33.000 Stacks
%    - Travis CI: 2. 6.800 Stacks
%    - Circle CI: 3. 7.350 Stacks
%    - GitHub Actions:
%    - GitLab CI: 4. 1590 Stacks
%    - AWS CodePipeline: 9. 228 Stacks
%    - CodeShip IO: 5. 1030 Stacks
%    - Azure DevOps: 201 Stacks