\subsubsection{CI/CD Herausforderungen}\label{ci_cd_herausforderungen}

Die Implementierung einer CI/CD Pipeline ist aufwendig und erfordert die Unterstützung einer ganzen Reihe von Stakeholdern des Unternehmens.
\footnote{Abbass et. al, vgl.~\cite{Abbass2019}~[S.2]}

\begin{itemize}
    \item \textbf{Komplexität:}
    Die Automatisierung von bestimmten Abläufen oder das Testen von Frontends kann komplex sein.
    Die Verwaltung der CI/CD-Server mit ihren Plugins ist ein nicht zu unterschätzender Aufwand für das Unternehmen.
    \footnote{Shahin et al., vgl.~\cite{Shahin2017}~[S.3910 - S. 3921]}

    \item \textbf{Langsame Pipelines:}
    Gerade, wenn viele Schritte in einer CI/CD Pipeline durchgeführt werden, kann diese Pipeline mehrere Minuten in Anspruch nehmen.
    Der Entwickler erhält das Feedback zu seinen Änderungen erst nach Abschluss der Pipeline.
    Bei Fehlschlägen kann das zu Frustration führen.
    \footnote{Abbass et. al, vgl.~\cite{Abbass2019}~[S.3]}

    \item \textbf{Kunden die kein CI/CD möchten:}
    Es gibt Kunden und Umgebungen in denen Updates nicht so einfach möglich oder erlaubt sind.
    Nicht alle Kunden sind zufrieden damit, neue Versionen zu erhalten.
    Veränderungen von Software können bei Ihnen z.B. neue Schulungen oder Zertifizierungen erforderlich machen.
    Im Umfeld von Embedded Applikationen ist CI/CD oft nicht möglich.
    \footnote{Shahin et al., vgl.~\cite{Shahin2017}~[S.3910 - S. 3912]}

    \item \textbf{Angst vor Veränderung:}
    Durch CI/CD und Automatisierung können Tätigkeiten, die vorher manuell durchgeführt wurden, entfallen.
    Dieser Wegfall verändert die Rollen und Tätigkeiten von Abteilungen.
    Die Angst vor Veränderung kann hier zu einer Ablehnung führen.
    \footnote{Shahin et al., vgl.~\cite{Shahin2017}~[S.3912]}

\end{itemize}


%    - Problems of CD: Large commits, slow integration, slow tests, ambigious test results.
%    - Continuous software engineering: Quick feedback from customer, short and frequent release cycles, improve sw quality, increase team productivity, includes automatic builds and tests
%    - Continuous Integration: widely established, merge and integrate frequently
%    - Continuous Delivery: Always production ready, fully automating quality checks and tests in production like environments, should reduce deployment risk, lower costs.
%    - Continuous Deployment: Automatic deployment into production, no manual steps like in CDE
%    - a minimum number of rules and regulation, e.g. for technology choices.
%    Traditional silo- organized IT departments with highly specialized knowledge need to be reorganized.
%    Cross-functional (service-centric) teams with general knowledge about the
%    service should be integrated within the IT department.
%    \footnote{Wiedemann, vgl.~\cite{Wiedemann2019}~[S.165]}
%    - The authors also found that not all customers are happy to receive new functionality on a continuous basis
%    and applying CD in the context of embedded systems is a challenge
%    - Based on 50 primary studies,it has been revealed that moving towards CD necessitates
%    significant changes in a given organization,
%    for example,team mindsets, organization’s way of working, and quality assurance activities are subject to change.
%    \footnote{Shahin et al., vgl.~\cite{Shahin2017}~[S.3912]}
