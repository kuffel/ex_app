\subsubsection{Auswahlkriterien}\label{iac_kriterien}

Zur Auswahl des Tools werden die folgenden Fragestellungen beantwortet:

\begin{itemize}
    \item \textbf{Art des Tools:}
    Handelt es sich bei dem Tool um Configuration Management oder Provisioning?

    \item \textbf{Behandlung von Infrastruktur:}
    Wird Infrastruktur als Mutable oder Immutable betrachtet und verwaltet?

    \item \textbf{Architektur:}
    Werden Änderungen auf die Infrastruktur geschoben (Push) oder von einem Client auf der Infrastruktur heruntergeladen (Pull)?

    \item \textbf{Paradigma:}
    Welches Paradigma wird in dem Code für die Infrastruktur verwendet?
    Gängige Paradigmen sind hier: Prozedural, Imperativ und Deklarativ

    \item \textbf{Master/Masterless:}
    Gibt es für dieses IaC-Tool einen Masterserver, der die Verwendung zentral steuert und überwacht oder verfolgt das Tool einen dezentralen Ansatz?

    \item \textbf{Agent/Agentless:}
    Muss auf der verwalteten Infrastruktur eine Software in Form eines Agents installiert werden?

    \item \textbf{Docker Unterstützung:}
    Können Docker Container verwaltet werden?

    \item \textbf{Programmiersprache:}
    In welcher Skript- bzw. Programmiersprache werden die Skripte verfasst?

    \item \textbf{Reifegrad:}
    Seit wann gibt es das Tool und in welcher Version steht es zur Verfügung?

    \item \textbf{Mischen von Tools:}
    Welche anderen Tools können zusätzlich verwendet werden?

    \item \textbf{Beliebtheit auf GitHub:}
    Wieviele Sterne, Forks und Beteiligte hat das Tool auf GitHub?

    \item \textbf{Tools des gleichen Anbieters:}
    Welche anderen Tools rund um IaC und DevOps sind von dem gleichen Anbieter verfügbar?

\end{itemize}
