\paragraph{Azure}\label{caas_tools_azure}

Azure wurde 2010 von Microsoft als Windows Azure gestartet und ist ein IaaS, Paas und SaaS Anbieter.
Die Dienste können hier auch lokal, hybrid, Edge oder Multicloud genutzt werden.
Immer mehr Desktop Produkte von Microsoft werden als SaaS Lösungen angeboten.
Diese sind dann oft günstiger als die lokale Variante.
\footnote{{Office 365 Plans, vgl.~\cite{AZURE_OFFICE_PLANS}}} \\

Die Stärke von Azure liegt in der Unterstützung von lokalen, hybriden und Edge Cloud Lösungen.
Da viele Unternehmen Windows Server einsetzen, kann mit Azure eine On-Premise Lösung mit der Cloud als Backup und Kapazitätserweiterung erschaffen werden. \\

Microsofts Umsatz an Cloud Diensten wie Azure und Office 365 hat sich von 2014 bis 2020 verdoppelt.
\footnote{{Microsoft's revenue from 2012 to 2020 financial years, by segment, vgl.~\cite{AZURE_REVENUE}}} \\

% `(https://www.datamation.com/cloud-computing/aws-vs-azure-vs-google-cloud-comparison.html)`
% Azure hat auch mehr Zertifizierungen als alle anderen Cloud Anbieter `(https://docs.microsoft.com/de-de/compliance/regulatory/offering-home)`

Azure hat ebenfalls eine Vielzahl an verschiedenen Diensten.
Dieser Abschnitt erläutert die wichtigsten Begriffe, die für die Einführung in Azure notwendig sind:

\begin{itemize}
    \item \textbf{Region und Availability Zone:}
    Azure hat das gleiche Konzept wie AWS, eine Region besteht ebenfalls immer aus mehreren AZs.
    \footnote{{Regions and Availability Zones in Azure, vgl.~\cite{AZURE_REGIONS_AND_AZS}}}

    \item \textbf{Virtual Machines:}
    Bis zu 416 vCPUs und 12 TB RAM, die per API oder über die Management-Konsole gestartet werden können.
    \footnote{{Virtual Machines, vgl.~\cite{AZURE_VIRTUAL_MACHINES}}}

    \item \textbf{Identity and Access Management:}
    Azure verwendet ein Active Directory mit Single-Sign-On und der Möglichkeit Unternehmenskonten für die Authentifizierung und Autorisierung zu verwenden.
    Eigene Anwendungen können mittels AD abgesichert werden.
    \footnote{{Azure Active Directory, vgl.~\cite{AZURE_AD}}}

    \item \textbf{Container Registry:}
    Von Azure verwaltete Container Registry und Zugriffsregeln lassen sich per AD einstellen.
    \footnote{{Container Registry, vgl.~\cite{AZURE_CONTAINER_REGISTRY}}}

    \item \textbf{Load Balancer:}
    Azure bietet hier zwei Produkte: Einerseits das Application Gateway für HTTP(S) Load Balancing und SSL Terminierung,
    andererseits den Azure Load-Balancer für den reinen TCP/UDP Traffic.
    \footnote{{Application Gateway, vgl.~\cite{AZURE_APPLICATION_GATEWAY}}}
    \footnote{{Load Balancer, vgl.~\cite{AZURE_LOAD_BALANCER}}}

    \item \textbf{Virtual Network:}
    Erlaubt - wie AWS VPCs - isolierte Netwerke in Azure zu erstellen und zu verwalten.
    \footnote{{Virtual Network, vgl.~\cite{AZURE_VIRTUAL_NETWORK}}}

\end{itemize}

% ###### 3.4.2.2.2. Container as a Service Angebot

Azure bietet einen \textbf{Kubernetes Service (AKS)} an,
welcher auch in Azure DevOps integriert werden kann, um CI/CD zu erleichtern.
Wie bei AWS ist es auch möglich On-Demand, Reserved und Spot Instances zu verwenden.
Azure bietet eine vollständig durch Azure verwaltete Kubernetes API an.
\footnote{{Azure Kubernetes Service, vgl.~\cite{AZURE_KUBERNETES}}}