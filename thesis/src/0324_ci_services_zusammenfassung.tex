\subsubsection{Zusammenfassung}\label{ci_services_zusammenfassung}

Alle SaaS Angebote sind sich sehr ähnlich.
Die Pipelines werden bei allen mit YAML-Dateien definiert.
Jedes der untersuchten Tools unterstützt Docker Container und das Injizieren von Passwörtern.
Die besonderen Features machen bei den untersuchten Tools den Unterschied aus. \\

Hier sind die Features von Circle CI besonders interessant, da die Nutzung von Orbs das Erstellen von komplexeren Build Pipelines deutlich vereinfacht.
Durch Orbs lassen sich die Installation von CLI Tools in einem Schritt erledigen.
Diese Installation erfordert in anderen Pipelines kleinere Bash Skripte. \\

Das kostenpflichtige Docker Layer Caching beschleunigt Builds deutlich und ist bei den anderen Tools in dieser Form nicht verfügbar. \\

Wenn man nur die SaaS-Lösungen betrachtet, bietet Circle CI das flexibelste Abrechnungssystem mit.
Man kann durch das Buchen von Credits nur so viele Build Minuten kaufen, wie auch aktuell erforderlich sind.
