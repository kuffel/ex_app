\paragraph{Ansible}\label{iac_tools_tools_ansible}

Ansible gehört zu RedHat und soll IT-Administrationsaufgaben automatisieren.
Mit Ansible lassen sich Befehle auf einer Vielzahl von Servern parallel ausführen, um sich wiederholende Tätigkeiten zu vermeiden.
\footnote{{Why Ansible, vgl.~\cite{ANSIBLE_OVERVIEW}}} \\

Das Tool arbeitet dabei \textsl{agentenlos}, d.h. auf den zu verwaltenden Servern muss nur SSH oder WinRM unterstützt werden.

Die \textsl{Ansible Automation Platform} besteht dabei aus folgenden Produkten:
\footnote{{Ansible Products, vgl.~\cite{ANSIBLE_PRODUCTS}}}

\begin{itemize}
    \item \textbf{Ansible Engine:}
    Das Kommandozeilen Tool mit dem YAML basierende Skripte auf den zu verwaltenden Servern ausgeführt werden.

    \item \textbf{Ansible Tower:}
    Dashboards, Job Scheduling und grafisches Tool für die Ausführung von Ansible Engine basierenden Tasks.

    \item \textbf{Ansible Analytics:}
    Analyse, Aggregation und Reporting von automatisierten Prozessen.

    \item \textbf{Content Collections:}
    Vorgefertigte Module, um Aufgaben mit Ansible zu lösen.

    \item \textbf{Automation Hub:}
    Suche nach vorgefertigten Modulen aus der Content Collection.

    \item \textbf{Automation services catalog:}
    Katalog, um eigene self service Lösungen auf Basis von Ansible im Unternehmen anzubieten.
\end{itemize}
