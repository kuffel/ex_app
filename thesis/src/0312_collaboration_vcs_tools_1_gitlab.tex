\paragraph{GitLab}\label{collaboration_gitlab}

GitLab wurde 2013 als Open-Source Software für die Installation auf dem eigenen Server gestartet. \footnote{thenextweb.com, vgl.~\cite{GITLAB_ANOUNCMENT}}
Die Lösung ist dreigeteilt in eine gehostete Version, eine kommerzielle und eine Open-Source Version. \footnote{GitLab Website, vgl.~\cite{GITLAB_ABOUT}}
Laut eigenen Angaben verwenden ca.
30 Millionen Entwickler und 100.000 Organisationen GitLab.
GitLab ist vor allem als selbsgehosteter Git Server beliebt, da sich die Open-Source Version auch für Unternehmen kostenlos nutzen lässt. \footnote{GitLab Pricing, vgl.~\cite{GITLAB_PRICING}}

Die kostenlose Version von GitLab hat bereits einen Funktionsumfang, der für kleine bis mittlere Unternehmen ausreichend ist.
Es lassen sich hier auch Kanban Boards, Milestones und das Ticket System nutzen.
GitLab stellt eine DevOps-Platform dar bei der auch mit GitLab Runner ein CI/CD System integriert.