\paragraph{Google Cloud Platform}\label{caas_tools_gcp}

Die Google Cloud Platform (GCP) wurde 2008 gestartet und läuft auf derselben Infrastruktur wie alle anderen Google Dienste.
\footnote{{Warum Google Cloud?, vgl.~\cite{GCP_WHY}}} \\

Google bietet fast 100 verschiedene Dienste an.
Darunter auch BigQuery, CDNs, und serverlose Dienste wie Cloud Functions.
\footnote{{Google Cloud products, vgl.~\cite{GCP_PRODUCTS}}} \\

Der Umsatz der Google Cloud hat sich von 2017 mit 4 Milliarden Dollar auf fast 9 Milliarden Dollar im Jahre 2019 mehr als verdoppelt.
\footnote{{Global Google Cloud revenues from 2017 to 2020, vgl.~\cite{GCP_REVENUE}}} \\

Hinter AWS und Azure ist GCP der drittgrößte Anbieter von Public Cloud Dienstleistungen.

% ###### 3.4.2.3.1. Einführung GCP

Auch GCP bietet eine Vielzahl verschiedener Dienste an.
Dieser Abschnitt erläutert die wichtigsten Begriffe, die für die Einführung in GCP notwendig sind:

\begin{itemize}
    \item \textbf{Regionen und Zonen:}
    Wie bei Amazon und Azure werden auch Regionen und Zonen für die Standorte verwendet.
    GPC ist zurzeit in 24 Regionen verfügbar, jede Region hat dabei mindestens 3 AZs.
    \footnote{{Regions and zones, vgl.~\cite{GCP_REGIONS_AND_ZONES}}}

    \item \textbf{Computing:}
    In diesem Bereich hat GCP mehrere Angebote, von AppEngine (vollkommen serverlos) bis hin zu VmWare Engine (Ausführung von VMs auf von GCP verwalteten VmWare Hosts).
    Das Pendant zu EC2 ist bei GCP die sogenannter Compute Engine.
    \footnote{{Compute Engine, vgl.~\cite{GCP_COMPUTE}}}
    % `(https://cloud.google.com/products/\#section-5)`

    \item \textbf{IAM:}
    Das GCP-Angebot ist ähnlich zu dem Angebot von AWS.
    Es lassen sich Rollen und Rechte zuweisen, die dann den Zugriff auf alle Ressourcen in GCP regeln.
    \footnote{{Cloud Identity and Access Management, vgl.~\cite{GCP_IAM}}}

    \item \textbf{Container Registry:}
    Ähnlicher Funktionsumfang wie ECR und Azure Docker Registry.
    Zusätzlich mit der Möglichkeit, Images auf Sicherheitslücken in Linux Paketen zu scannen.
    \footnote{{Container Registry, vgl.~\cite{GCP_REGISTRY}}}

    \item \textbf{Load Balancing:}
    GCP hat 3 verschiedene Load Balancer für HTTP(S), TCP und UDP.
    \footnote{{Cloud Load Balancing, vgl.~\cite{GCP_LB}}}

    \item \textbf{VPC:}
    Erlaubt, wie die Lösungen von AWS und Azure, isolierte Netzwerke in GCP zu erstellen.
    \footnote{{Virtual Private Cloud (VPC), vgl.~\cite{GCP_VPC}}}

\end{itemize}

% ###### 3.4.2.3.2. Container as a Service Angebot

Da Google Kubernetes ursprünglich erschaffen hat, kommt bei GCP ausschließlich Kubernetes zum Einsatz.
Der Dienst heißt Google Kubernetes Engine (GKE) wird vollständig von Google verwaltet und erlaubt die Nutzung von On-Demand, Reserved und Spot Instances.