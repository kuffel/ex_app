\newpage
\subsubsection{Auswahlkriterien}\label{collaboration_vcs_kriterien}

Zur Auswahl des Tools werden die folgenden Fragestellungen beantwortet:

\begin{itemize}
    \item \textbf{Unterstützung Privater Repositories:}
    Erlaubt das Tool private Repositories zu verwalten, die nicht öffentlich einsehbar sind?

    \item \textbf{Authentifizierungsmethoden:}
    Gerade im Unternehmensumfeld sind LDAP Integration oder andere Methoden wünschenswert.
    Welche Methoden für Authentifizierung und Autorisierung werden unterstützt?

    \item \textbf{Issue Tracker/Ticket System:}
    Unterstützt das Tool die Erstellung von Tickets und kann es diese in Boards verwalten?
    Tools mit einem Ticket System können ggf. direkt auch für die Planung von Meilensteinen und Scrum verwendet werden.

    \item \textbf{Code Editor:}
    Kann Quellcode direkt über die Web UI bearbeitet werden?

    \item \textbf{Code Review Unterstützung:}
    Lassen sich Merge Requests erstellen und über die Weboberfläche bearbeiten?
    
    \item \textbf{Code Review Approvals:}
    Können mehrere Personen einen Code Review durchführen, sodass dieser erst nach Bestätigung von mehreren Personen akzeptiert werden kann?

    \item \textbf{CI/CD Integration:}
    Ist ein Tool für die Verwaltung von CI/CD Pipelines integriert?

    \item \textbf{Self Hosting:}
    Kann das Tool \textsl{On-Premise} bzw. auf eigenen Servern installiert werden?

    \item \textbf{Kommerzieller Support:}
    Gibt es eine kommerzielle Version oder Support für Unternehmen?

    \item \textbf{Statistiken:}
    Unterstützt das Tool die Erfassung von Statistiken und Metriken zu den Aktivitäten?

    \item \textbf{Kosten:}
    Wie ist die Preisgestaltung?

    \item \textbf{Rang auf Stackshare:}
    Welchen Rang und wieviel Stacks hat das Tool auf Stackshare?
\end{itemize}