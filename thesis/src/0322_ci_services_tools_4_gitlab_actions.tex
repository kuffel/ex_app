\paragraph{GitLab Actions}\label{ci_services_tools_gitlab_actions}

GitLab Actions wurde 2013 als Teil von GitLab 3.0 eingeführt und ist seitdem Bestandteil von GitLab. \footnote{GitLab 3.0 Release Announcement, vgl.~\cite{GITLAB_ACTIONS}}

Die Preisgestaltung folgt hier ebenfalls den Preisen von GitLab.
Wenn GitLab auf eigenen Servern betrieben wird, können sogenannte GitLab Runner verwendet werden.
Diese laufen dann entweder auf dem gleichen System wie GitLab selbst oder auf anderen Servern (ähnlich wie bei Jenkins).

Die Runner sind Docker Container, d.h. es kann jedes Betriebssystem auf dem Docker Container ausgeführt werden können verwendet werden.

Wie üblich verwendet auch GitLab eine YAML DSL zur Definition der Build Skripte.

Passwörter und andere Geheimnisse werden bei GitLab über Umgebungsvariablen in den jeweiligen Build injiziert.