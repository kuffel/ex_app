\paragraph{Chef}\label{iac_tools_tools_chef}

% https://www.chef.io/products/enterprise-automation-stack

Chef wurde Anfang 2009 für Configuration Management gestartet. \footnote{{Announcing Chef, vgl.~\cite{CHEF_ANNOUNCEMENT}}}
Bei Chef werden sogenannte Rezepte verwendet, um IT-Ressourcen zu beschreiben und entsprechend zu konfigurieren. \\
Im Jahr 2020 wurde Chef von der Firma Progress übernommen. \footnote{{Progress Announces Acquisition of Chef, vgl.~\cite{CHEF_PROGRESS}}}
Chef verwendet einen Masterserver und auf den zu verwaltenden Server laufen Chef Clients. \\

Chef besteht aus mehreren Produkten mit unterschiedlichen Aufgaben, die als \textit{Chef Automation Stack} zusammengefasst sind. \footnote{{Chef Products, vgl.~\cite{CHEF_PRODUCTS}}}

\begin{itemize}
    \item \textbf{Chef Infrastructure Management:}
    Masterserver, der die Änderungen auf der Infrastruktur sicherstellt.

    \item \textbf{Chef App Delivery:}
    Ausrollen von Anwendungen.

    \item \textbf{Chef Compliance:}
    Compliance und Sicherheitsüberwachung.

    \item \textbf{Chef Desktop:}
    Verwaltung von Endgeräten, wie Windows PCs.
\end{itemize}
